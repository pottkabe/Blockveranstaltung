\section{Vorwort}

Sollten die Visionen der Fahrzeugbauer Realität werden, werden Fahrzeuge zu vollvernetzten Supercomputern, die
den menschlichen Insassen vollautonom an sein Ziel bringen.

Selbst wenn diese Vision noch einige Zeit in der Zukunft liegt, sind moderne Fahrzeuge bereits heute
leistungsfähige vernetzte Systeme, die mehr Rechenleistung haben als die Apollo Rakete, mit der die USA noch
vor wenigen Jahrzehnten auf dem Mond gelandet sind.

Ziel dieser Arbeit ist es einen groben Überblick über das Rechenzentrum Auto zu geben.\\

Beginnend mit einer allgemeinen Beschreibung der Vernetzung der verschiedenen Systeme in einem modernen Fahrzeug 
in Kapitel 2, über die konkret verwendeten Bussysteme in Kapitel 3  hin
zu den Sensoren und Steuergeräten, die dadurch verbunden werden, in Kapitel 4 und Kapitel 5 .

In Kapitel 6 werden noch die verschiedenen Assistenzsysteme beschrieben, auf der Grundlage dieses Systemverbundes bereits heute
realisiert werden können, während Kapitel 7  einen Ausblick gibt, welche Technologien und Entwicklungen, den Vorstellungen
der Forscher und Entwickler nach, in Zukunft auf uns warten.\\\\

\begin{center}
    \begin{tabular}{l l}
        \textbf{Autoren} &\\\\
        Peter Burger & Kapitel 5, 6\\
        Hoffmann Malte & Kapitel 3 (Automotive Ethernet, MOST, Bluetooth)\\
        Lay Andreas & Kapitel 3 (CAN, LIN, FlexRay)\\
        Schlauch Tobias & Kapitel 4\\
        Pottkamp Benjamin & Kapitel 7 \& Editorial\\
        Wiest Tobias & Kapitel 2
    \end{tabular}
\end{center}

