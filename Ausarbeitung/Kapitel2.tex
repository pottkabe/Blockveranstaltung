\section{Vorwort}

Sollten die Visionen der Fahrzeughersteller Realität werden, werden Fahrzeuge bald zu vollvernetzten Supercomputern, die
den menschlichen Insassen vollautonom an sein Ziel bringen.

Selbst wenn diese Vision noch einige Zeit in der Zukunft liegt, sind moderne Fahrzeuge bereits heute
hochleistungsfähige, vernetzte Systeme, die über mehr Rechenleistung verfügen als die Apollo Rakete, mit der die USA noch
vor wenigen Jahrzehnten auf dem Mond gelandet ist.
\\
Ziel dieser Arbeit ist es, einen groben Überblick über das Rechenzentrum Auto zu geben.\\

Beginnend mit einer allgemeinen Beschreibung der Vernetzung der verschiedenen Systeme in einem modernen Fahrzeug 
in Kapitel 2, über die konkret verwendeten Bussysteme in Kapitel 3, hin
zu den Sensoren und Steuergeräten, die dadurch verbunden werden, in Kapitel 4 und Kapitel 5, sollen zuerst die Grundstruktur und die vernetzten Teilsysteme
erläutert werden.

In Kapitel 6 werden dann die verschiedenen Assistenzsysteme moderner Fahrzeuge beschrieben, die auf der Grundlage dieses Systemverbundes bereits heute
realisiert werden können.

Kapitel 7 gibt schließlich einen Ausblick darüber, welche Technologievisionen und Entwicklungen, nach den Vorstellungen der Forscher und Entwickler, in Zukunft zu erwarten sind.\\\\

\begin{center}
    \begin{tabular}{l l}
        \textbf{Autoren} &\\\\
        Peter Burger & Kapitel 5, 6\\
        Hoffmann Malte & Kapitel 3 (Automotive Ethernet, MOST, Bluetooth)\\
        Lay Andreas & Kapitel 3 (CAN, LIN, FlexRay)\\
        Schlauch Tobias & Kapitel 4\\
        Pottkamp Benjamin & Kapitel 7, Abstract, Vorwort \& Editorial\\
        Wiest Tobias & Kapitel 2
    \end{tabular}
\end{center}

