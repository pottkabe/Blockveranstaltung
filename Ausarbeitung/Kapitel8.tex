\section{Ausblick}

    Neue Entwicklungsziele im Bereich des teil- und vollautonomen Fahrens stellen die elektronischen Fahrzeugsysteme vor neue Herausforderungen.\\
    Die selbstständige Durchführung der Fahraufgabe durch ein elektronisches System oder die aktive Unterstützung eines menschlichen Fahrzeugführers
    bedingt eine umfangreiche Erfassung von Fahrzeug- und Umgebungsdaten.\\
    Zu diesem Zweck müssen die bisherigen Sensorsysteme um weitere Systeme erweitert werden, die optische Umweltdaten über Kameras liefern, Umgebungsscans
    mittels Radar, Lidar oder Ultraschall durchführen und die erfassten Daten verarbeitet werden. ~\cite{.BP06}\\

    Langfristig ist eines der Hauptziele der Industrie und der Rechtsgeber eine umfassende Vernetzung
    sämtlicher Entitäten, die am Verkehrsgeschehen aktiv oder passiv partizipieren zu einem intelligenten
    Transport System (ITS).\\
    Ein solches ITS soll den Teilnehmern innovative Dienste anbieten, mit dem Ziel das Verkehrsgeschehen
    effizient zu verwalten und zu koordinieren sowie zusätzliche Sicherheit für alle Beteiligten zu bieten. ~\cite{.BP04}\\

    Vorraussetzung, um ein solches ITS aufzubauen ist eine Standardisierung der verwendeten Protokolle und Technologien.
    Zu diesem Zweck gibt es auf nationaler und internationaler Ebene verschiedene Organisationen und Konsortien,
    die dieses Ziel seit mehreren Jahren aktiv vorantreiben.\\

    \subsection{Vehicle to Everyhing}
    Grundlage eines ITS ist die Vehicle to Everything Kommunikation, die das Fahrzeug als zentralen Punkt in den
    Kontext der unterschiedlichen Umgebungssysteme und Entitäten setzt. Dabei wird genauer unterschieden in die Teilbereiche:
    
    \subsubsection{Vehicle to Vehicle}
    V2V beschreibt die Vernetzung der Fahrzeuge untereinander mit dem Ziel relevante Fahrzeugdaten z.Bsp. über RIchtung und Geschwindigkeit auszutauschen.
    Desweiteren kann über V2V Kommunkikation weitere sicherheitsrelevante Nachrichten aus anderen Teilbereichen weitergeleitet werden.

    \subsubsection{Vehicle to Network}
    V2N beschreibt die Vernetzung des Fahrzeugs mit dem Telekommunikationsnetz und der Cloud. Dadurch können über den
    rein lokalen Kontext hinaus Ressourcen genutzt und Daten geteilt werden. Ein Beispiel für eine solche V2N ANwendung, die bereits
    im Einsatz ist, ist die Integration von cloudbasierten Navigationslösungen wie Google Maps in das Fahrzeug.
    
    \subsubsection{Vehicle to Infrastructure}
    V2I beschreibt die Vernetzung des Fahrzeugs mit der umgebenden Verkehrsinfrastruktur. Zum Beispiel könnten smarte
    Verkehrsampeln einen Fahrzeuge über die verbleibende Wartezeit informieren oder die Ampelzyklen in Abhängigkeit
    der Anzahl der jeweils wartenden Fahrzeuge anpassen, um den Verkehrsfluss zu optimieren.

    \subsubsection{Vehicle to Pedestrian}
    V2P beschreibt die Vernetzung des Fahrzeugs mit nichtmotorisierten Verkehrsteilnehmern. Ziel ist explizit der
    Schutz dieser Verkehrsteilnehmer, die bei Unfällen einen inherenten Nachteil haben. V2P ist dabei jedoch allgemeiner
    zu verstehen und umfasst neben der aktiven Kommunikation der Entitäten auch die Erfassung von Fussgängern über rein
    fahrzeugseitige Sensorsysteme.

    \subsubsection{Vehicle to Device}
    V2D beschreibt die Kommunikation des Fahrzeugs mit elektronischen Geräten. Aktuelle Einsatzgebiete für diese Form der Kommunikation
    sind mobile Applikationen für Smartphones, mit denen Fahrzeugfunktionen von außerhalb gesteuert werden können. Aktuelle Beispiele
    sind die App von Tesla, mit der Fahrzeuge ausgeparkt werden können oder eine neue Entwicklung von Volvo mit der physische Schlüsseltechnologie
    durch eine mobile Smartphoneapp zuerst ergänzt und später dann ersetzt werden soll. ~\cite{.BP05}
     
    \subsubsection{Vehicle to Grid}
    V2G bescchreibt die Vernetzung eines elektrischen Fahrzeuges mit dem Stromnetz mit dem Ziel die Batterien elektrischer Fahrzeuge als Speichermedien
    bidirektional in das Stromnetz zu integrieren.

    \subsection{Technologien für V2X}
    
