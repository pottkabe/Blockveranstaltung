\section{Anhang}

\subsection{Quellenbegründung}

\subsubsection{Quellen Wiest, Tobias}

Digital networks in the automotive vehicle \cite{leen1999digital}

Das Werk wurde im Rahmen des Computing \& Control Engineering Journal Volume: 10 durch die IET veröffentlicht. Die britische IET ist nach der amerikanischen IEEE die zweitgrößte Ingenierusvertretung der Welt. Ebenso wie die IEEE fordert die IET höchste Anforderungen für Publikationen in Form von Peer-Reviews, Screenings, sowie vielen weiteren strengen Richtlinien und Anforderungen, sodass das Werk als vertraulich und qualitativ hochfertig anzusehen ist.\\\\

Bosch Autoelektrik und Autoelektronik \cite{reif2011bosch}

Das Buch wurde durch Ingnieure der Bosch-Entwicklungsabteilung, sowie von Mitarbeitern aus weiteren Unternehmen und Hochschulen erarbeitet, sodass das vermittelte Fachwissen als vertrauenswürdig und inhaltlich korrekt anzusehen ist.\\\\

In-vehicle networking: Protocols, challenges, and solutions \cite{TW_huang2018vehicle}

Das Werk wurde im Rahmen von IEEE Network Volume: 33 durch die IEEE veröffentlicht.
Neben Paper-Auswahlverfahren und Peer-Reviews gibt es strenge Richtlinien und Anforderungen, welche die Qualität einer IEEE Veröffentlichung sichern.\\\\

Gateway framework for in-vehicle networks based on CAN, FlexRay, and Ethernet \cite{TW_kim2014gateway}

Der Artikel wurde im Rahmen der IEEE Transactions on Vehicular Technology durch die IEEE veröffentlicht. Artikel in diesem Jounral müssen höchste Standards erfüllen. Neben Paper-Auswahlverfahren und Peer-Reviews gibt es strenge Richtlinien und Anforderungen, welche die Qualität einer IEEE Veröffentlichung sichern.\\\\

Automotive ethernet: in-vehicle networking and smart mobility \cite{hank2013automotive}

Das Werk wurde auf der Konferenz 2013 Design, Automation \& Test in Europe Conference \& Exhibition (DATE) durch die IEEE veröffentlicht. Die Standards einer derartigen Veröffentlichung durch die IEEE sind enorm hoch. Neben Paper-Auswahlverfahren und Peer-Reviews gibt es strenge Richtlinien und Anforderungen, welche die Qualität einer IEEE Veröffentlichung sichern.\\\\

Security in automotive bus systems \cite{wolf2004security}

Die Veröffentlichung stammt von der ESCRYPT GmbH, einem führenden Anbieter für IT-Security-Lösungen . Als 100\%ige Tochter der ETAS GmbH gehört ESCRYPT zur Bosch-Unternehmensgruppe, dem weltweit größten Automobilzulieferer. Aus diesem Grund, sowie den verwendeten, fachlich qualitativen Quellen, kann das Werk als vertrauenswürdig und inhaltlich korrekt angesehen werden.\\\\


\subsubsection{Quellen Pottkamp, Benjamin}

Car costs - automotive electronics costs worldwide 2030 \cite{BP02}

Als einer der weltweit führenden Anbieter sind die Statistiken, die über Statista bezogen werden können hochqualitativ.\\\\

Bosch Autoelektrik und Autoelektronik \cite{reif2011bosch}

Das Buch wurde durch Ingenieure der Bosch-Entwicklungsabteilung, sowie von Mitarbeitern aus weiteren Unternehmen und Hochschulen erarbeitet, sodass das vermittelte Fachwissen als vertrauenswürdig und inhaltlich korrekt anzusehen ist.\\\\

Next generation radar sensors in automotive sensor fusion systems \cite{BP06}

Dieses Paper wurde im Rahmen der Konferenz Sensor Data Fusion: Trends, Solutions, Applications durch die IEEE veröffentlicht. Die gesicherte Qualität,
die sich aus dem Peer-Review Prozeß und den allgemeinen Anforderungen der IEEE an die Paper ergeben, stellen sicher, dass die Informationen
aktuell waren zum Zeitpunkt der Veröffentlichung. An den in dem Paper beschriebenen  Anforderungen hat sich in der kurzen Zeit seit der Veröffentlichtung nichts grundlegendes
geändert, so dass die Aktualität nach wie vor gegeben ist.\\\\

Directive 2010/40/EU of the European Parliament \cite{BP04}

Direktive des Europäischen Parlaments bezüglich der Umsetzung von intelligenten Transportsystemen.\\\\

Intelligent transport systems standards \cite{BP11}

Umfassende Ausarbeitung zu dem Thema ITS, die insbesondere die verschiedenen Standards und Normen umfasst. Der Autor beschäftigt sich seit 1991 intensiv mit 
dem Thema und die direkte Referenzierung auf Normen und Standards bedeuten, dass ein repräsentatives Fakten- und Meinungsbild abgebildet wird.\\\\

Die Keyless-Technologie von Volvo: Komfort ohne Schlüssel \cite{BP05}

Blogbeitrag auf der Seite des Fahrzeeugherstellers Volvo über neue Technologieentwicklungen.\\\\

Optimizing the performance of vehicle-to-grid (V2G) enabled battery electric vehicles through a smart charge scheduling model \cite{BP08}

Artikel, der 2015 im International Journal of Automotive Technology einer internationalen Fachzeitschrift veröffentlicht wurde. Strikte Submission Richtlinien, die sowohl
Originalität als auch Peer Review fordern, garantieren hohe Qualität der veröffentlichten Artikel.\\\\

Intelligent Transport Systems: Communications access for land mobiles (CALM) - Architecture \cite{BP09}

ISO Standard\\\\

IEEE Standard 802.11p \cite{BP10}

IEEE Standard\\\\

\subsubsection{Quellen Lay, Andreas}

Controller area networks performance parameters measurement \cite{LA_CAN1}\\
Genehmigtes Patent, gültig bis 15.11.2027\\\\

Research and application of CAN and LIN bus in automobile Network System \cite{LA_CAN2}\\
Veröffentlicht durch IEEE. Referenziert Papers von IEEE-Konferrenzen.
International Conference on Control Automation and Systems.\\\\

Bosch Autoelektrik und Autoelektronik \cite{reif2011bosch}\\
Erfinder von CAN und CAN-FD, zudem Vertreiber von CAN,LIN,FlexRay-Knoten
und Steuereinheiten. War Teil des FelxRay und des LIN Konsortiums.\\\\

CAN with Flexible Data-Rate \cite{LA_CAN_FD1}\\
Erfinder von CAN und CAN-FD, zudem Vertreiber der aktuellen CAN-Generation.\\\\

The Frame Packing Problem for CAN-FD \cite{LA_CAN_FD2}\\
Veröffentlicht durch IEEE. Autoren besitzten viele Publikationen
bei der IEEE, aus dem selben Themenbereich. Werden von einigen anderen
Publikationen referenziert.\\\\

Comparison of FlexRay and CAN-Bus for Real-Time Communication \cite{LA_FR1}\\
Beziehen sich auf die Spezifikationen von FlexRay und CAN. 
Behauptungen stimmen mit der Spezifikation überein.\\\\

An analysis of CAN-based steer-by-wire system performance in vehicle \cite{LA_CAN3}\\
Veröffentlicht durch IEEE. Referenzieren hauptsächlich Publikationen von IEEE-Konferenzen.
Autoren besitzt viele Publikationen im selben Themenbereich.\\\\

Application of LIN Bus in Vehicle Network \cite{LA_LIN1}\\
Veröffentlicht durch IEEE. 
Referenzieren Datasheets der eingesetzten Controller und die Protokoll Spezifikationen.\\\\

LDF-Tool \cite{LA_LDF-Tool}\\
Vertreiber eines Konfigurationtool für LIN. Unter deren Kunden sind viele
Automobilhersteller und zuliefer.\\\\

Stra{\ss}enverkehrs-Zulassungs-Ordnung: StVZO \cite{LA_StVZO38}\\
Quelle wird von dem Bundesamt für Justiz betrieben.\\\\

Stra{\ss}enverkehrs-Zulassungs-Ordnung: StVZO \cite{LA_StVZO41}\\
Quelle wird von dem Bundesamt für Justiz betrieben.\\\\

Physical Layer Extraction of FlexRay Configuration Parameters \cite{LA_FR2}\\
Veröffentlicht durch IEEE. Autor hat viele Publikationen bei der IEEE,
im selben Themenbereich. Referenzieren die FlexRay Spezifikation.\\\\

A qualitative comparison of FlexRay and Ethernet in vehicle networks \cite{LA_FR3}\\
Veröffentlicht durch IEEE. Referenzieren andere IEEE Publikationen sowie
Spezifizierung von Ethernet und FlexRay. Haben viele Publikationen, in ähnlichen
Themengebieten, bei der IEEE veröffentlicht. Werden von einigen anderen Publikationen
referenziert.\\\\

\subsubsection{Quellen Schlauch, Tobias}

Zu den eigen geschrieben Texten hatte ich bereits durch meine Ausbildung und privatem Interesse schon einiges Vorwissen, welches ich durch die vorliegenden Quellen nochmals bestätigen wollte.
Bei den Artikeln handelt es sich um Informationen aus seriösen Quellen, da die Hersteller diese bereitgestellt und veröffentlicht haben, oder gar bei der Erstellung mitgeschrieben haben. .


Titel \cite{TS_sensor_aufteilen}\\
Die allgemeine Unterteilung der Sensoren ist aus Vorwissen. Die Quelle gab graphisch eine vollständige und sinnvolle Übersicht in Tabellenform.\\\\

Titel \cite{TS_sensoren}\\
Einteilung in mechanisch und nicht mechanische Sensoren, stammen aus eigenem Vorwissen und zur Bestätigung wurde diese Quelle verwendet.\\\\

Titel \cite{TS_temp_pic}\\
Die Darstellung wurde von dem Entwickler direkt zur Verfügung gestellt.\\\\

Titel \cite{TS_temp}\\
Eindeutige, detaillierte und gute Erklärung der Funktionsweise des Temperatursensors. Die Quelle basiert auf der Veröffentlichung des Herstellers. \\\\

Titel \cite{TS_see_beck}\\
Erklärung der Funktionsweise des See-Beck-Effekts. Dies wurde in Zusammenarbeit mit Physikern erläutert.\\\\

Titel \cite{TS_ind_sensor}\\
Bild eines Induktionssensors, welches die Funktion bildlich darstellt.\\\\

Titel \cite{TS_ind_funkt}\\
Seriöse Erklärung der Funktionsweise eines Induktionssensors. Der Verfasser des Textes hat selbst Informationen aus erster Hand und aus Kooperationen mit Autoherstellern. \\\\

Titel \cite{TS_ind_funkt_pic}\\
Bild wurde vom Hersteller direkt zur Verfügung gestellt. \\\\

Titel \cite{TS_oel}\\
Bosch hat die Informationen über diesen Sensor dargestellt und der Verfasser des Textes hat diese Information verwendet.\\\\

Titel \cite{TS_dreh}\\
Bosch und Opel hat die Funktionsweise über diesen Sensor offengelegt und der Verfasser hat aus diesen Informationen seinen Text erstellt.\\\\

Titel \cite{TS_dreh_pic_mag}\\
Bosch erstellte und veröffentlichte in Bildform die magnetroresistiv Funktionsweise des Sensors.\\\\

Titel \cite{TS_dreh_pic_photo}\\
Bosch erstellte und veröffentlichte in Bildform die photoelektrischen Funktionsweise des Sensors.\\\\

Titel \cite{TS_regen}\\
BMW und Bosch erstellten einen Artikel zu diesem Sensor.\\\\

Titel \cite{TS_regen_pic}\\
BMW und Bosch stellten die Funktionsweise dies Sensors in bildlicher Darstellung bereit.\\\\

Titel \cite{TS_swt_pic}\\
Der Reifenhersteller Continental publizierte eine Darstellung der Funktion des Sensors.\\\\

Titel \cite{TS_swt}\\
Continental beschrieb seinen eigen hergestellten Sensor sehr detailliert.\\\\

Titel \cite{TS_l_q_k_pic}\\
Der Autor hat seine Informationen aus erster Hand und diese Informationen wurden verwendet.\\\\

Titel \cite{TS_reifen}\\
Der Reifensensorhersteller Goodyear hatte mit einem Forschungszentrum diesen Sensor und dessen Funktion veröffentlicht.\\\\

Titel \cite{TS_rdks}\\
Diverse Autohersteller verfassten einen Artikel, aus welchem die Informationsquelle gebildet wurde.\\\\

Titel \cite{TS_rdks_pic}\\
Diverse Autohersteller stellten die Funktion graphisch dar.\\\\

Titel \cite{TS_hall}\\
Bosch und andere Autohersteller beschäftigten sich mit diesem Thema und die Informationen zu dem Artikel wurde aus diesen Quellen erzeugt.\\\\

Titel \cite{TS_hall_pic}\\
Bosch erstellte eine bildliche Darstellung des Sensors.\\\\

Titel \cite{TS_drehzahl_sensor}\\
Audi und Bosch verfassten einen Artikel, welcher diesen Sensor beschreibt. \\\\

Titel \cite{TS_drehzahl_sensor_pic}\\
Audi und Bosch stellten die Funktion bildlich dar.\\\\

Titel \cite{TS_lambda}\\
Die TU Darmstadt, sowie NGK beteiligten sich indirekt an dem Artikel.\\\\

Titel \cite{TS_lambda_pic}\\
Die TU Darmstadt stellte das Bild zur Verfügung. Es wurde in Zusammenarbeit mit dem Hersteller erstellt.\\\\

Titel \cite{TS_airbag}\\
Mehrere Autohersteller beteiligten sich an der Verfassung eines Artikels über diesen Sensor.\\\\

Titel \cite{TS_ultraschall_pic}\\
Das Bild kommt direkt aus einer Veröffentlichung von Bosch.\\\\

Titel \cite{TS_kamera_pic}\\
Leifiphysik ist eine seriöse Quelle, in welcher die Artikel von Professoren oder gar der Industrie verfasst werden.\\\\

\subsubsection{Quellen Burger, Peter}

Antiblockiersystem \cite{antiblockiersys.PB1}\\
Dies Quelle wurde gewählt, da Bosch, TRW und ATE die Hersteller von Antiblockiersystemen sind. Sie wurden in diesem Artikel als Hersteller genannt.\\\\

Das ESP \cite{ESP.PB1}\\
Diese Quelle wurde gewählt, da die Informationen von Bosch, Volkswagen, BMW und Mercedes sind. Und diese bekannte Automarken in Deutschland sind.\\\\

Transmission Control Module \cite{transmissioncontrol.PB1}\\
Dies wurde als Quelle, da es ein Foliensatz eine Hochschule ist und von einem Professor erstellt wurde.\\\\

Engine Control Module \cite{enginecontrol.PB1}\\
Diese Quelle wurde gewählt das die Informationen von einer Autorin kommen die 15 Jahre Erfahrung in der Autobranche hat. \\\\

Human-Machine Interface \cite{HMI.PB2}\\
Die Quelle stammt aus einer Firma die in solchen Bereichen spezialisiert ist.\\\\

Powertrain Control Module \cite{PCM.PB1}\\
Die Quelle wurde genommen, da sie speziell Informationen über die Motorbranche bietet.\\\\

Telematic Control Unit \cite{telematiccontrol.PB1}\\
Diese Quelle wurde genommen das die Firma Halbleiter herstellt und ihren Schwerpunkt in der Mobilität hat\\\\

Transmission Control Unit \cite{transmissioncontrol.PB2}\\
Hier ist Bosch der Betreiber der Webseite. Bosch ist eine bekannte Firma in Deutschland und hat großen Einfluss auf die Autobranche. Deshalb wurde hier die Quelle verwendet.\\\\

ECU \cite{ECU.PB6}\\
Hier ist Bosch der Betreiber der Webseite. Bosch ist eine bekannte Firma in Deutschland und hat großen Einfluss auf die Autobranche. Deshalb wurde hier die Quelle verwendet.\\\\

ECU \cite{ECU.PB1}\\
Hier ist Bosch der Betreiber der Webseite. Bosch ist eine bekannte Firma in Deutschland und hat großen Einfluss auf die Autobranche. Deshalb wurde hier die Quelle verwendet.\\\\

Battery Management System \cite{BMS.PB1}\\
Wikipedia wurde hier verwendet um eine groben Überblick über dieses Thema zu bekommen.\\\\

Body Control Module \cite{BCM.PB1}\\
Hier ist Bosch der Betreiber der Webseite. Bosch ist eine bekannte Firma in Deutschland und hat großen Einfluss auf die Autobranche. Deshalb wurde hier die Quelle verwendet.\\\\

Battery Management System \cite{BMS.PB2}\\
Bei dieser Quelle konnte man keine konkrete Beweise finden wer der Verfasser ist, jedoch stellt diese Webseite viele Informationen bezüglich Energie und Batterien von Firmen zur Verfügung.\\\\

ECU \cite{ECU.PB2}\\
Dieser Bericht wurde von einem Motorjournalist geschrieben und wurden somit als Referenz und Quelle benutzt.\\\\

ECU \cite{ECU.PB3}\\
Quelle wurde von einem Autor verfasst der 30 Jahre Erfahrung in der Autobranche besitzt und wurde deshalb herangezogen.\\\\

Seat Control Unit \cite{seatcontrol.PB1}\\
Die Informationen stammen von einer Firmenseite, die ihren Bereich beim Smarten Fahren hat und wurden deshalb als Quelle herangezogen.\\\\

Door Control Unit \cite{doorcontrol.PB1}\\
Die Informationen stammen von einer Firmenseite, die ihren Bereich beim Smarten Fahren hat und wurden deshalb als Quelle herangezogen.\\\\

Electronic Power Steering \cite{PSCU.PB1}\\
Diese Informationen sind von einem Post, wo sich keine genaue Informationen über den Autor befinden. Jedoch wurden in den Kommentaren keine negativen Bewertungen geschrieben.\\\\

Beranfahrassistent \cite{berganfahr.PB1}\\
Diese Seite wurde von ADAC geprüft und bekam Top Bewertungen. Da ADAC eine weitverbreitete Organisation in der Autobranche ist wurde dies als legitime Quelle angesehen.\\\\

Überholassistent \cite{ueberholassi.PB1}\\
Diese Quelle wurde lediglich als Cross-Referenz für andere Quellen benutzt.\\\\

Spurwechselassistent \cite{spurwechsel.PB1}\\
Die Quelle wurde von TÜV mit sehr gut bewertet und somit kann man davon ausgehen, dass diese Quelle keine falschen Informationen verbreitet.\\\\

Spurhalteassistent \cite{spurhalte.PB1}\\
Die Quelle wurde von TÜV mit sehr gut bewertet und somit kann man davon ausgehen, dass diese Quelle keine falschen Informationen verbreitet.\\\\

Parkassistent \cite{parkassi.PB1}\\
Die Quelle wurde von TÜV mit sehr gut bewertet und somit kann man davon ausgehen, dass diese Quelle keine falschen Informationen verbreitet.\\\\

Müdigkeitserkennung \cite{muedigkeitsassi.PB1}\\
Die Quelle wurde von TÜV mit sehr gut bewertet und somit kann man davon ausgehen, dass diese Quelle keine falschen Informationen verbreitet.\\\\

Notbremsassistent \cite{notbremsassi.PB1}\\
Die Quelle wurde von TÜV mit sehr gut bewertet und somit kann man davon ausgehen, dass diese Quelle keine falschen Informationen verbreitet.\\\\

Assistenzsysteme \cite{assistenzsysteme.PB1}\\
Die Quelle wurde von TÜV mit sehr gut bewertet und somit kann man davon ausgehen, dass diese Quelle keine falschen Informationen verbreitet.\\\\

Verkehrszeichenerkennung \cite{verkehrszeichenerk.PB1}\\
Diese Quelle stammt direkt aus dem Bussgeldkatalog, der von der Deutschenbehörde geschrieben wird. Daher kann man davon ausgehen, 
dass dieser Beitrag auf fundiertem Wissen basiert.\\\\

ISA \cite{ISA.PB1}\\
Die Informationen stammen von der Eurpäischen Kommision und daher von einer öffentlichen Behörde. Es wurde auch aktuell updated und 
ist somit auf dem neusten Stand.\\\\

Berganfahrassistent \cite{berganfahr.PB2}\\
Der Beitrag stammt von der offiziellen Webseite von Seat einem Autohersteller. Daher kann man davon ausgehen, dass diese Informationen
Gewichtung haben.\\\\

Kreuzungsassistent \cite{kreuzungsassi.PB1}\\
Die Quelle wurde von dem Autor Wolgang Sievernich geschrieben, der Redakteur bei dem Auto- und Reiseclub Deutschland e.V. ist. Diese Informationen sollten 
also ein gewisses Gewicht haben.\\\\

Spurhalteassistent \cite{spurhalte.PB2}\\
Hier ist Bosch der Betreiber der Webseite. Bosch ist eine bekannte Firma in Deutschland und hat großen Einfluss auf die Autobranche. Deshalb wurde hier die Quelle verwendet.\\\\

Assistenzsysteme \cite{assistenzsysteme.PB2}\\
Die Quelle ist der ADAC und dieser ist eine weitverbreitete Organisation in der Autobranche und bei vielen anerkannt.\\\\

Müdigkeitserkennung \cite{muedigkeitsassi.PB2}\\
Hier ist Bosch der Betreiber der Webseite. Bosch ist eine bekannte Firma in Deutschland und hat großen Einfluss auf die Autobranche. Deshalb wurde hier die Quelle verwendet.\\\\

Speed Limiter \cite{speedlimiter.PB1}\\
Die Quelle ist ein Online-Shop für den Kauf eines Autos, deshalb sollten hier auch richtige Angaben gemacht werden und von Personen geschrieben sein, die etwas von 
ihrem Handwerk verstehen.\\\\

Verkehrszeichenerkennung \cite{verkehrszeichenerk.PB2}\\
Diese Seite wurde von ADAC geprüft und bekam Top Bewertungen. Da ADAC eine weitverbreitete Organisation in der Autobranche ist wurde dies als legitime Quelle angesehen.\\\\

Stauassistent \cite{stauassi.PB1}\\
Die Informationen stammen von der offiziellen Homepage des Automarke Volkswagen, diese ist eine weitverbreitete Automarke und hat somit fundiertes Wissen.\\\\

Reverse Detection \cite{reversedetection.PB1}\\
Die Quelle stammt von der Homepage der Firma Geotab, die im Bereich der Fuhrparkmanagementtechnologie arbeit. Aus diesem Grund kann man davon ausgehen, dass sie 
tiefreichendes Wissen über Fahrzeuge verfügen.\\\\

Parkassistent \cite{parkassi.PB2}\\
Die Quelle ist der ADAC und dieser ist eine weitverbreitete Organisation in der Autobranche und bei vielen anerkannt.\\\\

Kreuzungsassistent \cite{kreuzungsassi.PB2}\\
Die Informationen stammen von der offiziellen Homepage des Automarke Volkswagen, diese ist eine weitverbreitete Automarke und hat somit fundiertes Wissen.\\\\

ISA \cite{ISA.PB2}\\
Dieser Artikel wurde von zwei Personen verfasst, der eine hat einen Doktortietel und der zweite eine Masterabschluss. Man kann davon ausgehen das diese 
Personen richtige Informationen verbreiten.\\\\

LDW \cite{LDW.PB1}\\
Der Artikel wurde einem Bill Howard verfasst, über den keine näheren Informationen zu sehen sind. Man kann nur verfolgen das er mehrere Artikel über den Automobielbereich verfasst hat. 
Somit konnte man von einem Tieferenwissen ausgehen.\\\\

Bergabfahrassistent \cite{bergabfahr.PB1}\\
Die Informationen stammen von der offiziellen Homepage des Automarke Volvo, diese ist eine weitverbreitete Automarke und hat somit fundiertes Wissen.\\\\

Bremsassistent \cite{bremsassi.PB1}\\
Die Quelle ist ein Fachanbieter für Bremstechnik und muss somit fundiertes Wissen in diesem Gebiet haben. Aus diesem Grund wurde diese Quelle gewählt.\\\\

Abbiegeassistent \cite{abbiegeassi.PB1}\\
Die Informationen kommen direkt aus dem Bundesministerium für Verkehr und digitale Infrastruktur. Man kann davon ausgehen das die bereitgestellten Informationen richtig sind.\\\\

Assistenzsysteme \cite{assistenzsysteme.PB3}\\
Die Quelle stammt aus einer Webseite von einer Firma die Automobilteile liefert und hat somit ein gutes Kenntniss über die einzelnen Teile. \\\\

Notbremsassitent \cite{notbremsassi.PB2}\\
Diese Seite nimmt ihre Informationen von Bosch und Audi und deshalb kann man die Informationen als glaubwürdig betrachten\\\\

Fahrereassistenzsysteme \cite{Audi.PB1}\\
Die Informationen stammen direkt aus einem Audi Paper. Da Audi eine bekannte Marke in Deutschland ist  wurde sie als Quelle herangezogen.\\\\
\newpage
\subsubsection{Quellen Hoffmann, Malte}

Durch Vorlesungen (Verteilte Systeme und Elektronische Systeme im Automobil (AEI 1)) \\
habe ich bereits Vorwissen gehabt, mit denen ich meine Quellen überprüft habe 
und eigenes Wissen einbringen konnte.
\\

Ethernet in Fahrzeugen: Das Bussystem für moderne Automobile \cite{.MH_Ethernet}\
In dieser Quelle wurden mehrere, zur Zeit der Veröffentlichung aktuelle, andere
Publikationen referenziert. Darunter auch Standards, Masterarbeiten und vertrauenswürdige
Verfasser, wie das IEEE.
Zudem ist diese Arbeit an einer Universität unter Aufsicht eines Professors entstanden.\\\\

Automotive Ethernet: In-vehicle Networking and Smart Mobility \cite{.MH_Vehicle}\\
Diese Quelle wurde durch die IEEE veröffentlicht und hat ein Peer-Review durchgangen.
Die Arbeit wurde in mehreren anderen durch die IEEE veröffentlichten Arbeiten zitiert.
Zudem sind die Autoren bereits berufserfahren und haben meist einige andere Publikationen
in diesem Bereich veröffentlicht.\\\\

Intra-Vehicle Networks: A Review \cite{.MH_Review}\\
Diese Quelle wurde durch die IEEE veröffentlicht und durch Peer-Review bestätigt. 
Die Autoren haben bereits Berufserfahrung sammeln können und haben mehrere andere 
teilweise sehr erfolgreiche Publikationen im selben Themengebiet veröffentlicht. 
Zudem wurde die Quelle bereits von einigen anderen wissenschaftlichen Arbeiten zitiert.\\\\

IEEE Standard Bluetooth \cite{.MH_Blue2}\\
Der Standard ist direkt von IEEE veröffentlicht worden.\\\\

Bild zu Bustopologie \cite{.MH_Bus}\\
Das Bild zeigt eine Standarddarstellung der Bustopologie. Dies konnte ich 
durch mein Vorwissen im Bereich Bussysteme bestätigen.\\\\

Bild zu Sterntopologie \cite{.MH_Star}\\
Das Bild zeigt eine Standarddarstellung der Sterntopologie. Dies konnte ich 
durch mein Vorwissen im Bereich Bussysteme bestätigen.\\\\

Bild zu Baumtopologie \cite{.MH_Tree}\\
Das Bild zeigt eine Standarddarstellung der Baumtopologie. Dies konnte ich 
durch mein Vorwissen im Bereich Bussysteme bestätigen.\\\\

Bild zu MOST-Ringtopologie \cite{.MH_Ring}\\
Das Bild kenne ich in ähnlicher Form aus dem Skript zu AEI 1. Zudem deckt
es sich mit der in Fachliteratur beschriebenen Topologie und Einsatzweise.\\\\

Bild zu Scatternet \cite{.MH_Scatter}\\
Das Bild ist in ähnlicher Form auch im Standard (IEEE Standard Bluetooth) 
abgebildet und stimmt auch mit der Beschreibung im Standard überein.\\\\
