\section{Fahrerassistenzsysteme}
    \subsection{Einführung}
    Fahrerassistenzsysteme sind Systeme die elektrisch den Autofahrer in bestimmten
    Situationen helfen und unterstützen. Je nach Marke, Modell und Land können
    unterschiedliche Kombinationen von Fahrerassistenzsysteme hinzugebucht werden
    oder sind als Standardequipment bereits im Fahrzeug eingebaut. Solche Assistenten
    werden für die Sicherheit und den Komfort in die Automobile eingebaut. Für die
    einzelnen Systeme werden verschiedene Geräte wie Sensoren, Radar, Video oder
    Ultraschall verwendet, um ausreichend Informationen für die einzelnen Assistenten
    zu bekommen. Wenn nun eine riskante Situation aus der Sicht der Systeme entsteht,
    wird diese dem Fahrer durch visuelle und akustische Signale mitgeteilt. Solche
    Fahrerassistenten ersetzen bei manchen Stellen komplett den Fahrer und dies ist
    der Weg zum autonomen Fahren.
    ~\cite{assistenzsysteme.PB1} ~\cite{assistenzsysteme.PB2} ~\cite{assistenzsysteme.PB3}
    ~\cite{assistenzsysteme.PB4}
    \subsection{Fahrerassistenzsysteme}

        \subsubsection{Antiblockiersystem (ABS)}
        ABS (Antiblockiersystem) verhindert das bei einer Vollbremsung die Räder
        blockiert werden und der Fahrer die Kontrolle über das Fahrzeug verliert.
        Um die Kontrolle nicht zu verlieren, wird in kurzen Abständen immer wieder
        der Bremsvorgang gelöst.
        ~\cite{assistenzsysteme.PB2} ~\cite{antiblockiersys.PB1}

        \subsubsection{Elektronisches Stabilitätsprogramm}
        (Electronic stability control) / \textbf{ESP} (Elektronisches Stabilitätsprogramm)
        ist ein Sicherheitssystem zur Spur- und Stabilitätskontrolle des Fahrzeugs.
        Es soll als Unterstützung bei Ausweichmanövern oder beim Verlust der Kontrolle über das Fahrzeug dienen. 
        Zu diesem Zweck wird das Motormomentum reduziert und sollte dies nicht ausreichen leitet das System einen aktiven
        Bremsvorgang ein.
        ~\cite{assistenzsysteme.PB2} ~\cite{ESP.PB1}
        
        \subsubsection{Antriebsschlupfregelung (ASR)}
        Ein Assistent für die Unterstützung beim Anfahren oder Beschleunigen. Der
        Assistent sorgt für eine gleichmäßge Verteilung des Momentmum auf die Räder.
        Das Endziel soll verhindern, dass die Räder durchdrehen. Dadurch soll eine 
        bessere Fahrstabilität und Fahrtraktion entstehen. Unter anderem soll auch 
        die Fahrzeugführung und die Lenkung verbessert werden.
        ~\cite{assistenzsysteme.PB2} ~\cite{ASR.PB1} ~\cite{ASR.PB2} 

        \subsubsection{Bremsassistens (BAS)}
        Beim Bremsen sorgt dieser Assistent für eine Verstärkung des Bremsdruck, sodass
        eine Vollbremsung bei einer Notfallsituation entsteht. Der Assistent soll Unfälle 
        verhindern oder zumindest abschwächen, indem er den Fahrer bei dem Bremsvorgang 
        unterstützt.
        ~\cite{assistenzsysteme.PB2} ~\cite{bremsassi.PB1} ~\cite{bremsassi.PB2}

        \subsubsection{Berganfahrhilfe}
        Aktiviert eine automatische Handbremse, sodass beim Anfahren am Berg kein Rückwärtsrollen
        entsteht. Der Assistent soll das hektische Bremspedal lösen und Gaspedal treten 
        verhindern.
        ~\cite{berganfahr.PB1} ~\cite{berganfahr.PB2}  ~\cite{assistenzsysteme.PB2}
        
        \subsubsection{Bergabfahrhilfe (HDC)}
        Beim Berg Herabfahren regelt dieser Assistent die Geschwindigkeit bei steilen Hängen. 
        Ohne diesen Assistent regelt der Motor die Geschwindigkeit, wenn das Gaspedal nicht 
        betätigt wird. Ist jetzt aber das Automobil schwerer ist es nicht mehr möglich das 
        der Motor diese Arbeit selbständig erledigt und das HDC unterstützt den Motor durch 
        zusätzliches Bremsen.
        ~\cite{assistenzsysteme.PB2} ~\cite{bergabfahr.PB1} 

        \subsubsection{Abstandsregeltempomat (ACC, Adaptiv Cruise Control)}
        Der Assistent sorgt für die passende Geschwindigkeit, sodass der richtige Abstand
        zu dem vorausfahrenden Fahrzeug eingehalten wird. Wenn man zu nahe auf das vordere
        Fahrzeug auffährt, wird abgebremst und bei zu großem Abstand beschleunigt. Der Abstand
        den das Fahrzeug einhalten soll kann in dem Assistenten eigestellt werden, manche Fahrzeuge
        haben die Möglichkeiten von kurz, mittel und groß. Wird von dem Assistent kein vorausfahrendes
        Fahrzeug erkannt, so arbeitet der Assistent als Geschwindigkeitsregeler für den Fahrer. Der
        neue Assistent soll zusätzlich das Abbremsen bis zum Stillstand und Stop\&Go beherrschen.
        ~\cite{assistenzsysteme.PB2} ~\cite{Audi.PB1}

        \subsubsection{Automatisches Notbremssysteme (AEBS)}
        Dieser Fahrerassistent soll selbständig mögliche Zusammenstöße erkennen. Wenn sich solch
        eine Möglichkeit bietet, soll der Assistent das Fahrzeug abbremsen lassen und somit ein
        Zusammenstoß verhindern. Dieser Assistent geht so weit, bis das Fahrzeug zum Stillstand 
        gebracht ist.
        ~\cite{notbremsassi.PB1} ~\cite{assistenzsysteme.PB1}  ~\cite{assistenzsysteme.PB2}
        ~\cite{notbremsassi.PB2}
        
        \subsubsection{Spurhalteassistent (LKA)}
        Ein System, das den Fahrer unterstützt eine optimale Spur auf der Fahrbahn beizubehalten.
        Das Fahrzeug soll die Position im Bezug zu der Spur- und Straßenbegrenzung halten. Die
        Spur wird durch leichte Lenkeingriffe bei dem Fahrzeug beibehalten. Der Fahrer kann aber
        jederzeit selber in das Fahrverhalten eingreifen.
        ~\cite{spurhalte.PB1} ~\cite{assistenzsysteme.PB1} ~\cite{spurhalte.PB2}  ~\cite{assistenzsysteme.PB2}

        \subsubsection{Spurverlassenswarner (LDW, Lane Departure Warning)}
        Eine Warnfunktion die dem Fahrer hilft und ihn warnt, wenn das Automobil die Fahrspur
        verlassen sollte. Voraussetzung für eine Warnung ist, dass der Fahrer keinen Richtungsblinker
        gesetzt hat und ein Fahrspurwechsel beabsichtigt. Der Blinker wird als Signal gesehen, ob das 
        Fahrzeug die Spur aktiv verlassen will oder ob es ein Abkommen von der Spur ist. Die 
        Warnung kann auf verschiedene Arten auftreten: Lenkeingriff, Lenkradvibration oder visuelles
        Signal.
        ~\cite{assistenzsysteme.PB2} ~\cite{LDW.PB1}

        \subsubsection{Überholassistent}
        Ein Assistent der dem Fahrer hilft ein Überholmanöver durchzuführen oder sogar ein
        komplettes Überholmanöver selber durchführt. Beim Überholmanöver ist der Assistent 
        behilflich, indem er dem Fahrer Informationen über den toten Winkel gibt. Dadurch 
        wird dem Fahrer geholfen, ob beim Verlassen der Spur von hinten ein Fahrzeug kommt 
        oder nicht. Der Überholassistenten wird auch Spurwechselassistent genannt.
        ~\cite{ueberholassi.PB1} ~\cite{spurwechsel.PB1} ~\cite{assistenzsysteme.PB1} 
        ~\cite{assistenzsysteme.PB2}
        
        \subsubsection{Fernlichtassistenten}
        Dieser Assistent bietet einen großen Sicherheitsgewinn. Er sorgt für eine bessere
        Ausleuchtung der Straße und kann bei neuen Modellen dynamisch eingesetzt werden.
        Diese Funktion bietet dem Fahrer, dass dieser sich nicht auf das Einschalten und
        Abblenden des Fernlicht konzentrieren muss. Dies wird von dem Assistent übernommen,
        dieser erkennt ob Gegenverkehr kommt und blendet ab und danach schaltet er das Fernlicht
        wieder ein.
        ~\cite{assistenzsysteme.PB2} ~\cite{Audi.PB1}

        \subsubsection{Speed Limiter}
        Der Speed Limiter ist eine Funktion, die eine Obergrenze für die Geschwindigkeit festlegt.
        Man kann bei dem Speed Limiter also eine Geschwindigkeit einstellen 
        und dann ist es egal wie stark man auf das Gaspedal drückt. Die Geschwindigkeit wird nicht 
        über die eingestellte Geschwindigkeit gehen. Es soll helfen die Geschwindigkeit besser 
        einzuhalten und kann auch mit der Verkehrszeichenerkennung kombiniert werden.
        ~\cite{assistenzsysteme.PB2} ~\cite{assistenzsysteme.PB2} 

        \subsubsection{Intelligent Speed Adaptation (ISA)}
        Das System soll den Fahrer unterstützen, indem es Rückmeldungen bei überhöhter Geschwindigkeit
        zurückgibt. Das System soll die Verkehrsbedingung und Straßenverhältnisse beurteilen und
        anhand diesen Informationen geeignete Rückmeldungen geben. Unter anderem wir die Position des 
        Fahrzeugs über das GPS festgestellt und somit dann auch das Geschwindigkeitslimit ermittelt.
        ~\cite{ISA.PB1}  ~\cite{speedlimiter.PB1} ~\cite{ISA.PB2}

        \subsubsection{Erkennung und Notbremsung beim Rückwärtsfahren \newline (Reversing detection)}
        Ein Assistent der Informationen über Objekte hinter dem Fahrzeug übermittelt. Diese Funktion
        wird beim Rückwärtsfahren benötigt, um keinen Zusammenstoß zu verursachen. Dem Fahrer wird 
        über akustische Signale mitgeteilt, ob sich hinter dem Auto ein Objekt befindet oder nicht.
        Im Notfall kann das System eingreifen und eine Notbremsung durchführen.
        ~\cite{assistenzsysteme.PB2}  ~\cite{reversedetection.PB1}

        \subsubsection{Abbiegeassistent}
        Es sollen Fußgänger, Fahrradfahrer und andere Objekte, die sich neben dem Automobil befinden
        oder näher kommen erkennen und davor warnen. Somit soll ein Unfall zwischen den Verkehrsteilnehmer
        verhindert werden. Die Warnung die dieser Assistent dem Fahrer bietet ist ein optisches Signal,
        welches am Seitenspiegel aufleuchtet.
        ~\cite{assistenzsysteme.PB2} ~\cite{abbiegeassi.PB1} ~\cite{abbiegeassi.PB2}

        \subsubsection{Müdigkeitserkennung und Aufmerksamkeitsüberwachung}

        Ein System das die Aufmerksamkeit des Fahreres kontrolliert. Das System soll den Fahrer durch
        bestimmte Methoden analysieren, ob der Fahrer noch wach ist. Sollte dies nicht der Fall sein,
        wird der Fahrer durch ein Signal gewarnt. Eine Methode zur Analyse ist das Lenkverhalten, jedoch
        unterscheiden sich die Methoden unter den Herstellern.
        ~\cite{muedigkeitsassi.PB1} ~\cite{assistenzsysteme.PB1}  ~\cite{assistenzsysteme.PB2}
        ~\cite{muedigkeitsassi.PB2}
        
        \subsubsection{Stauassistent}
        Wenn ein Fahrzeug in einen Stau gelangt, so kann der Stauassistent aktiviert werden. Sollte dieser
        aktiv sein, kann der Fahrer die Beschleunigung und das Bremsen dem System überlassen. Es wird
        jedoch verlangt, dass der Fahrer jederzeit in der Lage ist einzugreifen und das Fahrzeug übernehmen
        kann.
        ~\cite{stauassi.PB1} ~\cite{stauassi.PB2}
        
        \subsubsection{Notfallassistent}
        Der neue Audi besitzt einen Notfallassistenten, dieser soll das Auto bei einem Notfall übernehmen 
        und zum Stillsatnd bringen. Der Assistent überwacht ob die Hände des Fahrers am Lenkrad sind und 
        ob über eine gewisse Zeit das Gas- und Bremspedal betätigt wurde. Ist dies nicht der Fall, so 
        wird die Situation als Notfall kategorisiert und es werden bestimmte Maßnahmen eingeleitet.
        Zuerst wird versucht der Fahrer durch verschiedene Reize anzusprechen, dass er wieder die Kontrolle 
        über das Fahrzeug übernehmen kann. Sind diese Versuche erfolgslos, so werden Maßnahmen für den 
        Notfall eingeleitet. Diese Maßnahmen haben das Ziel das Fahrzeug sicher zum Stehen zu bringen, einen Zugang des 
        Notarzt zu gewährleisten und einen Notruf absetzen.
        ~\cite{Audi.PB1}
        
        \subsubsection{Parksystem}
        Einer der bekanntesten Assistenten ist der Parkassistent. Der Parkassistent kann optische und 
        akustische Signale zur Unterstützung beim Einparken bereitstellen. Die akustischen Signale werden
        durch die Ultraschallsensoren gesteuert, das heißt wenn man zu nah an ein Objekt fährt, desto lauter 
        wird das Signal. Das optische Signal wird durch eingebaute Kameras an bestimmten Stellen geliefert.
        In der Zwischenzeit wird sogar zwischen aktiven Parkassistenten und passiven Parkassistenten 
        unterschieden, beim passiven muss das Lenken selber übernommen werden und beim aktiven wird 
        dies von dem Assistenten übernommen.
        ~\cite{parkassi.PB1} ~\cite{assistenzsysteme.PB1} ~\cite{parkassi.PB2}
        
        \subsubsection{Kamerabasierte Verkehrszeichenerkennung}
        Oftmals wird vom Fahrer ein Verkehrszeichen übersehen oder nicht wahrgenommen. Mit der Verkehrszeichenerkennung 
        übernimmt eine Kamera im Fahrzeug diese Aufgabe für den Fahrer und kann ihn somit unterstützen.
        Eine große Hilfe ist die Verkehrszeichenerkennung bei Geschwindigkeitsbegrenzungen, bei denen der 
        Assistent die Schilder erfasst und den Fahrer warnt, wenn dieser zuschnell fährt. Der Warnhinweis 
        wird auf dem Display angezeigt und je nach Assistent verschwindet er nach einer gewissen Zeit oder 
        wenn die richtige Geschwindigkeit erreicht wurde.
        ~\cite{assistenzsysteme.PB1} ~\cite{verkehrszeichenerk.PB1} ~\cite{verkehrszeichenerk.PB2}
        
        \subsubsection{Kreuzungsassistent}
        Dieser Assistent soll Kollisionen mit Querverkehr verhindern. Der Kreuzungsassistent soll den Fahrer 
        unterstützen Objekte, die durch Gegenstände versteckt sind, zu erkennen. In solchen Situationen findet 
        man sich meistens an Kreuzungen wieder. Der Assistent soll zudem noch die Unaufmerksamkeit des 
        Fahrers kompensieren, wenn sich dieser auf andere Verkehrsteilnehmer konzentriert. Das Arbeiten  
        dieses Assisten geschieht durch Sensoren, die das Umfeld links und rechts neben dem Fahrzeug  im 
        Auge behalten. Der Kreuzungsassistent sollte immer eingeschaltet sein, sodass er in kritischen Situation 
        reagieren kann. Warnungen kann der Assisten über Anzeigen im Display als Bild, Ertönen eines Warnsignals
        und Aufblinken eines Symbold dem Fahrer übermitteln. Im Notfall wenn der Fahrer nicht 
        reagiert wird durch den Assistent einen Notbremseingriff betätigt.
        ~\cite{kreuzungsassi.PB1} ~\cite{kreuzungsassi.PB2}
