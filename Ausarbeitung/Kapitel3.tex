\section{Vernetzung im Fahrzeug}

1. EINLEITUNG Vernetzung
- übergang/einleitungs teil, begründung warum relevant

- unterteilung in 4 subsystems (paper) powertrain, chassis, body, infotainment
- Roughly speaking, an auto-electronics system
consists of four subsystems: powertrain, chassis,
body, and infotainment. Various protocols have
been developed for these systems.
- controller area network (CAN) has long been
used to transmit the majority of in-vehicle commu-
nication signals and is still widely deployed in the
powertrain and body control domains.
- FlexRay, with a distinguished determinism and fault-tol-
erance capability, is typically used in support of
advanced chassis control and communication
backbones.
- While the local interconnection net-
work (LIN) was designed for cost-saving purposes
and is often used in low-speed communications
- where high networking performance is usually
not required, Media Oriented Systems Transport
(MOST) networks are notably expensive and are
commonly used in premium vehicles as the carri-
er of infotainment data.

-Each sub-system has its own control units such
as mechanical, electrical, or computer controls,
which are independent of but cooperative with
those of other sub-systems.

- Powertrain: The powertrain sub-system refers
to a set of automobile components including
engine, transmission, shafts, wheels, and so on,
that generate power to the vehicle. The power-
train can also include sensors and actuators to
improve the comfort of the ride, reduce pollution
caused by exhaust systems, increase fuel efficien-
cy, and strengthen the vehicle’s safety.

- Chassis: The chassis sub-system refers to the
internal framework that supports the powertrain
and all other components, except the engine,
that are required for driving. Brakes, steering, and
suspension are commonly know components in
the chassis. Similar to the powertrain, sensors and
actuators can be installed in the chassis domain
and have stringent delay requirements.

- Body: The body sub-system includes com-
fort-controlling components such as climate con-
trol, seat adjustment, window rolling, lights, and
so on. Sensors for these components usually have
low bandwidth requirements and have a relatively
high-tolerance for delays (milliseconds).

Infotainment: The infotainment sub-system
provides an interface for facilitating the inter-
action between humans and the automobile’s
electronics. This sub-system presents information
acquired from sensors or ECUs in a user-friendly
and interactive manner to users for the purpose
of entertainment. Users’ mobile devices can also
be hooked up to the infotainment sub-system via
Bluetooth, WiFi, or cellular networks. Moreover,
as the infotainment sub-system controls different
parts of the vehicle and displays their information,
it is capable of communicating with other sub-sys-
tems. For example, some advanced infotainment
systems can remotely diagnose vehicle problems
by gathering diagnostic data from other sub-sys-
tems. As such, the infotainment system requires
high bandwidth but is tolerant to delay in the
scale of millisecond.

- SAFETY-SYSTEM: a sub-system that is vertical to these sub-systems
is the driver assistance and safety system. This
system is designed to assist drivers to operate a
vehicle in a safer manner. It includes built-in GPS,
cruise control, automatic parking, and even more
advanced functionalities such as lane-shift warn-
ing, collision avoidance, intelligent speed adjust-
ment or advice, driver drowsiness detection, and
blind spot detection. This system typically has its
own sensors and dedicated controllers, which
intensively interact with other systems (i.e., pow-
ertrain, chassis, body, and infotainment). Also,
it usually demands high bandwidth, and such a
demand keeps increasing due to more camer-
as (for example, BMW’s surround view feature)
being installed on new car models.

WIRED technologies:

CAN: CAN is an asynchronous serial bus network that intercon-
nects devices, sensors, and actuators in a system or a sub-system
for control applications.CAN is a multi-master communication protocol that is
designed for data integrity and automotive applications with data rates up to 1 Mb/s.
CAN is known for its low cost and high reliability. Due to these advantages, CAN is widely used
in the powertrain, chassis, and body electronics. CAN has relatively low bandwidth
and is a shared medium for data transmissions, which significantly restricts its application to other
domains such as infotainment.

LIN is a universal asynchronous receiver-transmitter-based,
single-master, multiple-slave networking protocol
that was purposefully developed for automotive
sensor and actuator networking applications. LIN
offers a cost-effective alternative for connecting
motors, switches, and sensors in the vehicle.
LIN is often used for body electronics as it is free and its bandwidth
requirement is easy to meet.

- FlexRay: FlexRay is a network communications
protocol with a dual-channel data rate up to 10 Mb/s
for advanced in-vehicle controls. The notable fea-
ture of FlexRay is its dual-channel architecture
that provides reassurance to satisfy the reliability
requirements of emerging safety systems such
as brake-by-wire.

- Media Oriented System Transport (MOST):
MOST is a high-speed multimedia network tech-
nology that is specially designed for the info-
tainment system.

- Ethernet: Ethernet has been the standard tech-
nology for local area networks ever since it was
invented, and plays a critical role in the develop-
ment of all types of communications. Automo-
tive Ethernet is the Ethernet technology when it
is used to connect components within a vehicle.
Being initially designed to meet various (e.g., elec-
tric, bandwidth, latency, synchronization, and net-
work management) requirements, the advantages
of automotive Ethernet are obvious:
• It increases the communication bandwidth
for advanced driving functionalities and the
infotainment system.
• It changes the in-vehicle network structures
from decentralized domain-specific topolo-
gies to hierarchical ones.
• It enhances the scalability and flexibility of
future in-vehicle networking architecture.
In addition to these networking technologies,
advancements in power and data transmissions
are also worth noting.

- Power Line Communication (PLC): PLC is a
set of technologies using electronic wires to simul-
taneously carry both data and electric power.
Traditionally, electronic devices in a vehicle were
always required to have at least two connections: 
one for data transmission and one for power supply.

WIRELESS: Wireless communication technologies are a
potential alternative for in-vehicle networking.
These technologies not only allow wireless con-
nections to be established between drivers/
passengers’ personal electronics and vehicles’
built-in infotainment system, but can eliminate the
need for wirlelines by also interconnecting sen-
sors, actuators, and ECUs in such a manner.

Anforderungen:
zeitkritisch, zuverlässig, redundant für x by wire systeme,
engere Anbindung, oft zeitgesteuerte systeme, da zeitliche aktualität 
der daten bestimmbar und ausbleiben einer nachricht sofort erkannt wird.
Aber nicht sehr gut erweiterbar, muss oft im voraus geplant werden. 
Composability ist wichtig -> Zusammensetzbarkeit, unabhängige Integration
von Teilsystemen in das Gesamtsystem. Überprüfung und Fehlerbehandlugn somit
auch auf das Teilsystem beschränkt.

Systemunterteilung:
powertrain, chassis, body, infotainment
Arten der Vernetzung:
wired: can, lin, flexray, most, ethernet
wireless: wifi, bluethooth, uwb, zigbee
topologien:
stern, maschen, ring, bus, hybrid
Adressierungsarten:
teilnehmerbasiert, nachrichtenorientiert, übertragungsorientiert
Buszugriffsverfahren:
tdma, cdma, master-slave, random, csma/c(a/d)
strukturierung:
osi referenzmodell
Steuermechanismen:
ereignis und zeitgesteuert

Vernetzung im KFZ:
früher einfache signalleitungen zur kom. Bedarf zu hoch, Lösung Bussysteme
viele signale in mehreren steuergeräten benötigt, somit sinnvoll gemeinsam genutzte
größen in einem steuergerät zu berechnen und über interne netzwerk kommunikation auszutauschen
beispiele (pre crash und acc s83)

vorteile von bussystemen gegenüber herkömmlichen verdrahteten signalleitungen:
kosten, gewicht,bauraum, höhere zuverlässigkeit, funktionssicherheit (geringere anzahl steckverbindungen),
vereinfachte fahrzeugmontage, durch sensor empfangene signale können auf bus gegeben und von
den relevanten komponenten empfangen und verarbeitet werden. Composability -> einfachere einbindung neuer
Systeme an einen Datenbus anstelle neuer verkabelung.

Anforderungen an bussysteme:
datenübertragungsrate, störsicherheit

klassifizierung von bussystemen:
klasse a,b,c,d

Einsatzgebiete:
antriebsstrang, chassis, innenraum, telematik
antriebsstrang u chassis -> primär echtzeitanwednungen (evtl details s86)
Innenraum -> Multiplexaspekte bei der Vernetzung. (details s86)
Telematik -> Multimedia- und Infotainmentanwendungen vernetzt (details s86)

kopplung der netzwerke:
- da protokolle nicht kompatibel -> gateway, liest empfangene daten ein
und passt format für das ziel netzwerkprotokoll an.
- entweder zentrales gateway oder verteilte gateways
- eingesetzte bussysteme sind herstellerabhängig


