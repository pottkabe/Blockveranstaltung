\section{Fahrerassistenzsysteme}

    \subsection{Einführung}
    Fahrerassistenzsysteme sind Systeme die elektrisch den Autofahrer in bestimmten
    Situationen helfen und unterstützen. Je nach Marke, Modell und Land können
    unterschiedliche Kombinationen von Fahrerassistenzsysteme hinzugebucht werden
    oder sind als Standardequipment bereits im Fahrzeug eingebaut. Solche Assistenten
    werden für die Sicherheit und den Komfort in die Automobile eingebaut. Für die
    einzelnen Systeme werden verschiedene Geräte wie Sensoren, Radar, Video oder
    Ultraschall verwendet, um ausreichend Informationen für die einzelnen Assistenten
    zu bekommen. Wenn nun eine riskante Situation aus der Sicht der Systeme entsteht,
    wird diese dem Fahrer durch visuelle und akustische Signale mitgeteilt. Solche
    Fahrerassistenten ersetzen bei manchen Stellen komplett den Fahrer und dies ist
    der Weg zum autonomen Fahren.
    \cite{assistenzsysteme.PB1}\cite{assistenzsysteme.PB2} \cite{assistenzsysteme.PB3}
    \cite{assistenzsysteme.PB4}

    \subsection{Fahrerassistenzsysteme}

        \subsubsection{Antiblockiersystem (ABS)}
        ABS (Antiblockiersystem) verhindert, dass bei einer Vollbremsung die Räder
        blockieren und der Fahrer die Kontrolle über das Fahrzeug verliert.
        Um die Kontrolle nicht zu verlieren, wird in kurzen Abständen immer wieder
        der Bremsvorgang gelöst.
        \cite{assistenzsysteme.PB2} \cite{antiblockiersys.PB1}

        \subsubsection{Elektronisches Stabilitätsprogramm}
        (Electronic stability control) / \textbf{ESP} (Elektronisches Stabilitätsprogramm)
        ist ein Sicherheitssystem zur Spur- und Stabilitätskontrolle des Fahrzeugs.
        Es soll als Unterstützung bei Ausweichmanövern oder beim Verlust der Kontrolle über das Fahrzeug dienen. 
        Zu diesem Zweck wird das Drehmoment des Motors reduziert, sollte dies nicht ausreichen leitet das System einen aktiven
        Bremsvorgang ein.
        \cite{assistenzsysteme.PB2} \cite{ESP.PB1}
        
        \subsubsection{Antriebsschlupfregelung (ASR)}
        Ein Assistent für die Unterstützung beim Anfahren oder Beschleunigen. Der
        Assistent sorgt für eine gleichmäßge Verteilung des Drehmoments auf die Räder.
        Das Endziel soll verhindern, dass die Räder durchdrehen. Dadurch soll eine 
        bessere Fahrstabilität und Fahrtraktion entstehen. Unter anderem soll auch 
        die Fahrzeugführung und die Lenkung verbessert werden.
        \cite{assistenzsysteme.PB2} \cite{ASR.PB1} \cite{ASR.PB2} 

        \subsubsection{Bremsassistent (BAS)}
        Beim Bremsen sorgt dieser Assistent für eine Verstärkung des Bremsdruck, sodass
        eine Vollbremsung bei einer Notfallsituation entsteht. Der Assistent soll Unfälle 
        verhindern oder zumindest abschwächen, indem er den Fahrer bei dem Bremsvorgang 
        unterstützt.
        \cite{assistenzsysteme.PB2} \cite{bremsassi.PB1} \cite{bremsassi.PB2}

        \subsubsection{Berganfahrhilfe}
        Aktiviert eine automatische Handbremse, sodass beim Anfahren am Berg kein Rückwärtsrollen
        entsteht. Der Assistent soll das hektische Bremspedal lösen und Gaspedal treten 
        verhindern.
        \cite{berganfahr.PB1} \cite{berganfahr.PB2}  \cite{assistenzsysteme.PB2}
        
        \subsubsection{Bergabfahrhilfe (HDC)}
        Beim Berg Herabfahren regelt dieser Assistent die Geschwindigkeit bei steilen Hängen. 
        Ohne diesen Assistenten regelt der Motor die Geschwindigkeit, wenn das Gaspedal nicht 
        betätigt wird. Bei höhrerem Gewicht des Automobils ist es nicht mehr möglich, dass 
        der Motor diese Arbeit selbständig erledigt, sodass der Motor durch zusätzliches Bremsen 
        des HDC unterstützt wird.
        \cite{assistenzsysteme.PB2} \cite{bergabfahr.PB1} 

        \subsubsection{Abstandsregeltempomat (ACC, Adaptiv Cruise Control)}
        Der Assistent sorgt für die passende Geschwindigkeit, sodass der richtige Abstand
        zu dem vorausfahrenden Fahrzeug eingehalten wird. Wenn man zu nahe auf das vordere
        Fahrzeug auffährt, wird abgebremst oder bei zu großem Abstand beschleunigt. Der Abstand
        den das Fahrzeug einhalten soll kann in dem Assistenten eingestellt werden, manche Fahrzeuge
        haben die Möglichkeiten von kurz, mittel und groß. Wird von dem Assistent kein vorausfahrendes
        Fahrzeug erkannt, so arbeitet der Assistent als Geschwindigkeitsregeler für den Fahrer. Der
        neue Assistent soll zusätzlich das Abbremsen bis zum Stillstand und Stop\&Go beherrschen.
        \cite{assistenzsysteme.PB2} \cite{Audi.PB1}

        \subsubsection{Automatisches Notbremssystem (AEBS)}
        Dieser Fahrerassistent soll selbständig mögliche Zusammenstöße erkennen. Wenn sich solch
        eine Möglichkeit bietet, soll der Assistent das Fahrzeug abbremsen lassen und somit einen
        Zusammenstoß verhindern. Dieser Assistent geht so weit, bis das Fahrzeug zum Stillstand 
        gebracht ist.
        \cite{notbremsassi.PB1} \cite{assistenzsysteme.PB1}  \cite{assistenzsysteme.PB2}
        \cite{notbremsassi.PB2}

        \subsubsection{Spurhalteassistent (LKA)}
        Ein System, das den Fahrer unterstützt eine optimale Spur auf der Fahrbahn beizubehalten.
        Das Fahrzeug soll die Position im Bezug zu der Spur- und Straßenbegrenzung halten. Die
        Spur wird durch leichte Lenkeingriffe bei dem Fahrzeug beibehalten. Der Fahrer kann aber
        jederzeit selber in das Fahrverhalten eingreifen.
        \cite{spurhalte.PB1} \cite{assistenzsysteme.PB1} \cite{spurhalte.PB2}  \cite{assistenzsysteme.PB2}

        \subsubsection{Spurverlassenswarner (LDW, Lane Departure Warning)}
        Eine Warnfunktion die dem Fahrer hilft und ihn warnt, wenn das Automobil die Fahrspur
        verlassen sollte. Voraussetzung für eine Warnung ist, dass der Fahrer keinen Richtungsblinker
        gesetzt hat und ein Fahrspurwechsel beabsichtigt. Der Blinker wird als Signal gesehen, ob das 
        Fahrzeug die Spur aktiv verlassen will oder ob es ein Abkommen von der Spur ist. Die 
        Warnung kann auf verschiedene Arten auftreten: Lenkeingriff, Lenkradvibration oder visuelles
        Signal.
        \cite{assistenzsysteme.PB2} \cite{LDW.PB1}

        \subsubsection{Überholassistent}
        Ein Assistent der dem Fahrer hilft ein Überholmanöver durchzuführen oder sogar ein
        komplettes Überholmanöver selber durchführt. Beim Überholmanöver ist der Assistent 
        behilflich, indem er dem Fahrer Informationen über den toten Winkel gibt. Dadurch 
        wird dem Fahrer geholfen, ob beim Verlassen der Spur von hinten ein Fahrzeug kommt 
        oder nicht. Der Überholassistent wird auch Spurwechselassistent genannt.
        \cite{ueberholassi.PB1} \cite{spurwechsel.PB1} \cite{assistenzsysteme.PB1} 
        \cite{assistenzsysteme.PB2}
