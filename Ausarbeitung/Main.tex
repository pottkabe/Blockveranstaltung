\documentclass[]{article}
\usepackage[english,german]{babel}
\usepackage{cite}
\usepackage{graphicx}
\usepackage{tabularx}
\usepackage{subcaption}
\usepackage[utf8]{inputenc}
\captionsetup[subfigure]{list=true, font=large, labelfont=bf, 
	labelformat=brace, position=top}


%opening
\title{Überblick über die elektronischen Systeme von Fahrzeugen}
\author{
	Peter Burger
	\and
	Malte Hoffmann
	\and
	Andreas Lay
	\and
	Benjamin Pottkamp
	\and
	Tobias Schlauch
	\and
	Tobias Wiest
}
\date{14. Februar 2020}

\begin{document}
\sloppy

\maketitle

\begin{abstract}
	Mittlerweile machen elektronische Systeme etwa ein Drittel der Gesamtkosten bei der Produktion von
	Personenkraftwagen aus \cite{BP01} Von Motorsteuerung, über aktive und passive Sicherheitssysteme, 
	Wartung und Diagnose bis hin zur Unterhaltungselektronik sind Personenkraftwagen inzwischen hochgradig
	vernetzte Systeme.\\
	
	Mit den aktuellen Entwicklungen in Richtung teil- und vollautonomer Systeme wird diese Vernetzung noch weiter zunehmen 
	und die elektronischen Systeme werden in absehbarer Zukunft Hauptwertträger eines Fahrzeugs werden.\\
	
	Ziel dieser Ausarbeitung ist es dem interessierten Leser einen Überblick über die wichtigsten elektronischen Systeme in
	modernen Fahrzeugen und deren Interaktion untereinander zu geben. Ein gewisses technisches Grundverständnis vorrausgesetzt 
	soll er in der Lage sein, neue Entwicklungen in den Kontext des aktuellen Stand der Technik zu setzen.\\
	
	Da es sich um ein komplexes Thema handelt, das auf beschränktem Platz dargeboten werden soll, müssen gewisse Teilbereiche naturgemäß
	kürzer ausfallen oder gänzlich ignoriert werden, weshalb diese Arbeit als Ausgangspunkt für eine nähere Beschäftigung mit dem Thema
	zu verstehen ist.
\end{abstract}

\tableofcontents

%including the separate Files
%\section{Vernetzung im Fahrzeug}

1. EINLEITUNG Vernetzung
- übergang/einleitungs teil, begründung warum relevant

- unterteilung in 4 subsystems (paper) powertrain, chassis, body, infotainment
- Roughly speaking, an auto-electronics system
consists of four subsystems: powertrain, chassis,
body, and infotainment. Various protocols have
been developed for these systems.
- controller area network (CAN) has long been
used to transmit the majority of in-vehicle commu-
nication signals and is still widely deployed in the
powertrain and body control domains.
- FlexRay, with a distinguished determinism and fault-tol-
erance capability, is typically used in support of
advanced chassis control and communication
backbones.
- While the local interconnection net-
work (LIN) was designed for cost-saving purposes
and is often used in low-speed communications
- where high networking performance is usually
not required, Media Oriented Systems Transport
(MOST) networks are notably expensive and are
commonly used in premium vehicles as the carri-
er of infotainment data.

-Each sub-system has its own control units such
as mechanical, electrical, or computer controls,
which are independent of but cooperative with
those of other sub-systems.

- Powertrain: The powertrain sub-system refers
to a set of automobile components including
engine, transmission, shafts, wheels, and so on,
that generate power to the vehicle. The power-
train can also include sensors and actuators to
improve the comfort of the ride, reduce pollution
caused by exhaust systems, increase fuel efficien-
cy, and strengthen the vehicle’s safety.

- Chassis: The chassis sub-system refers to the
internal framework that supports the powertrain
and all other components, except the engine,
that are required for driving. Brakes, steering, and
suspension are commonly know components in
the chassis. Similar to the powertrain, sensors and
actuators can be installed in the chassis domain
and have stringent delay requirements.

- Body: The body sub-system includes com-
fort-controlling components such as climate con-
trol, seat adjustment, window rolling, lights, and
so on. Sensors for these components usually have
low bandwidth requirements and have a relatively
high-tolerance for delays (milliseconds).

Infotainment: The infotainment sub-system
provides an interface for facilitating the inter-
action between humans and the automobile’s
electronics. This sub-system presents information
acquired from sensors or ECUs in a user-friendly
and interactive manner to users for the purpose
of entertainment. Users’ mobile devices can also
be hooked up to the infotainment sub-system via
Bluetooth, WiFi, or cellular networks. Moreover,
as the infotainment sub-system controls different
parts of the vehicle and displays their information,
it is capable of communicating with other sub-sys-
tems. For example, some advanced infotainment
systems can remotely diagnose vehicle problems
by gathering diagnostic data from other sub-sys-
tems. As such, the infotainment system requires
high bandwidth but is tolerant to delay in the
scale of millisecond.

- SAFETY-SYSTEM: a sub-system that is vertical to these sub-systems
is the driver assistance and safety system. This
system is designed to assist drivers to operate a
vehicle in a safer manner. It includes built-in GPS,
cruise control, automatic parking, and even more
advanced functionalities such as lane-shift warn-
ing, collision avoidance, intelligent speed adjust-
ment or advice, driver drowsiness detection, and
blind spot detection. This system typically has its
own sensors and dedicated controllers, which
intensively interact with other systems (i.e., pow-
ertrain, chassis, body, and infotainment). Also,
it usually demands high bandwidth, and such a
demand keeps increasing due to more camer-
as (for example, BMW’s surround view feature)
being installed on new car models.

WIRED technologies:

CAN: CAN is an asynchronous serial bus network that intercon-
nects devices, sensors, and actuators in a system or a sub-system
for control applications.CAN is a multi-master communication protocol that is
designed for data integrity and automotive applications with data rates up to 1 Mb/s.
CAN is known for its low cost and high reliability. Due to these advantages, CAN is widely used
in the powertrain, chassis, and body electronics. CAN has relatively low bandwidth
and is a shared medium for data transmissions, which significantly restricts its application to other
domains such as infotainment.

LIN is a universal asynchronous receiver-transmitter-based,
single-master, multiple-slave networking protocol
that was purposefully developed for automotive
sensor and actuator networking applications. LIN
offers a cost-effective alternative for connecting
motors, switches, and sensors in the vehicle.
LIN is often used for body electronics as it is free and its bandwidth
requirement is easy to meet.

- FlexRay: FlexRay is a network communications
protocol with a dual-channel data rate up to 10 Mb/s
for advanced in-vehicle controls. The notable fea-
ture of FlexRay is its dual-channel architecture
that provides reassurance to satisfy the reliability
requirements of emerging safety systems such
as brake-by-wire.

- Media Oriented System Transport (MOST):
MOST is a high-speed multimedia network tech-
nology that is specially designed for the info-
tainment system.

- Ethernet: Ethernet has been the standard tech-
nology for local area networks ever since it was
invented, and plays a critical role in the develop-
ment of all types of communications. Automo-
tive Ethernet is the Ethernet technology when it
is used to connect components within a vehicle.
Being initially designed to meet various (e.g., elec-
tric, bandwidth, latency, synchronization, and net-
work management) requirements, the advantages
of automotive Ethernet are obvious:
• It increases the communication bandwidth
for advanced driving functionalities and the
infotainment system.
• It changes the in-vehicle network structures
from decentralized domain-specific topolo-
gies to hierarchical ones.
• It enhances the scalability and flexibility of
future in-vehicle networking architecture.
In addition to these networking technologies,
advancements in power and data transmissions
are also worth noting.

- Power Line Communication (PLC): PLC is a
set of technologies using electronic wires to simul-
taneously carry both data and electric power.
Traditionally, electronic devices in a vehicle were
always required to have at least two connections: 
one for data transmission and one for power supply.

WIRELESS: Wireless communication technologies are a
potential alternative for in-vehicle networking.
These technologies not only allow wireless con-
nections to be established between drivers/
passengers’ personal electronics and vehicles’
built-in infotainment system, but can eliminate the
need for wirlelines by also interconnecting sen-
sors, actuators, and ECUs in such a manner.

Anforderungen:
zeitkritisch, zuverlässig, redundant für x by wire systeme,
engere Anbindung, oft zeitgesteuerte systeme, da zeitliche aktualität 
der daten bestimmbar und ausbleiben einer nachricht sofort erkannt wird.
Aber nicht sehr gut erweiterbar, muss oft im voraus geplant werden. 
Composability ist wichtig -> Zusammensetzbarkeit, unabhängige Integration
von Teilsystemen in das Gesamtsystem. Überprüfung und Fehlerbehandlugn somit
auch auf das Teilsystem beschränkt.

Systemunterteilung:
powertrain, chassis, body, infotainment
Arten der Vernetzung:
wired: can, lin, flexray, most, ethernet
wireless: wifi, bluethooth, uwb, zigbee
topologien:
stern, maschen, ring, bus, hybrid
Adressierungsarten:
teilnehmerbasiert, nachrichtenorientiert, übertragungsorientiert
Buszugriffsverfahren:
tdma, cdma, master-slave, random, csma/c(a/d)
strukturierung:
osi referenzmodell
Steuermechanismen:
ereignis und zeitgesteuert

Vernetzung im KFZ:
früher einfache signalleitungen zur kom. Bedarf zu hoch, Lösung Bussysteme
viele signale in mehreren steuergeräten benötigt, somit sinnvoll gemeinsam genutzte
größen in einem steuergerät zu berechnen und über interne netzwerk kommunikation auszutauschen
beispiele (pre crash und acc s83)

vorteile von bussystemen gegenüber herkömmlichen verdrahteten signalleitungen:
kosten, gewicht,bauraum, höhere zuverlässigkeit, funktionssicherheit (geringere anzahl steckverbindungen),
vereinfachte fahrzeugmontage, durch sensor empfangene signale können auf bus gegeben und von
den relevanten komponenten empfangen und verarbeitet werden. Composability -> einfachere einbindung neuer
Systeme an einen Datenbus anstelle neuer verkabelung.

Anforderungen an bussysteme:
datenübertragungsrate, störsicherheit

klassifizierung von bussystemen:
klasse a,b,c,d

Einsatzgebiete:
antriebsstrang, chassis, innenraum, telematik
antriebsstrang u chassis -> primär echtzeitanwednungen (evtl details s86)
Innenraum -> Multiplexaspekte bei der Vernetzung. (details s86)
Telematik -> Multimedia- und Infotainmentanwendungen vernetzt (details s86)

kopplung der netzwerke:
- da protokolle nicht kompatibel -> gateway, liest empfangene daten ein
und passt format für das ziel netzwerkprotokoll an.
- entweder zentrales gateway oder verteilte gateways
- eingesetzte bussysteme sind herstellerabhängig



%\include{Kapitel4a}
%\graphicspath{{./Images/Kapitel5/}}

\section{Sensoren}
	\subsection{Allgemeines} 
	
	Abbildung \ref{fig:TS01} zeigt eine grafische Übersicht der wichtigsten Sensoren in einem Fahrzeug:
			
	\begin{figure}[h!]
		\includegraphics[width=\textwidth] {sensor_uebersicht.png}
        \caption{Übersicht der Sensoren im Fahrzeug \cite{BP01}}
        \label{fig:TS01}
	\end{figure}	
	
		\subsubsection{Begriffsdefinition}
	
        Unter Sensor versteht man im allgemeinen eine
        \begin{itemize}
            \item Komponente
            \item Fühler
            \item Detektor
            \item Aufnehmer
        \end{itemize}
            der eine physikalische Größe oder chemische Effekte durch messtechnische Verfahren aufnimmt und in ein analoges, elektrisches Strom- / Spannungssgnal umwandelt.
        
        \subsubsection{Arten von Sensoren \cite{TS_sensor_aufteilen}}

        Grundlegend kann man zwischen mechanischen und nicht-mechanischen Sensoren unterscheiden. \\
        \begin{itemize}
            \item \textbf{Mechanische Sensoren}: verändern durch mechanische Einwirkungen von außen, zum Beispiel durch Kraft- oder Druckeinwirkung, ihre elektrische Eigenschaften.
            \item \textbf{Nicht-Mechanische Sensoren}: verändern ihre Eigenschaften durch nicht-mechanische Einwirkungen, zum Beispiel chemisch durch Lichteinwirkung.
        \end{itemize}
        \begin{table}
        \begin{tabularx}{\textwidth} {l|l|l}

                    \textbf{Sensorart} & \textbf{Messtechnik} & \textbf{Beispiele}\\
					\hline
                    \multicolumn{3}{l}{\textbf{Mechanische Sensoren}}\\
                    \hline
					Resistiv & Eletrische & Dehnungsmessstreifen \\
					&Widerstandsänderung& Potentiometrische Sensoren\\
					\hline
                    
                    Kapazitiv & Kapazitätsänderung & Drucksensor\\
                    && kapazitiver Näherungsschalter \\
					\hline
					
					Temperatur & Kontaktthermometrie & Thermoelement\\
					&& Widerstandsthermometer \\
                    \hline

                    \multicolumn{3}{l}{\textbf{Nicht-Mechanische Sensoren}}\\
                    \hline

                    Induktiv & Änderung der Induktion & Schwingungsaufnehmer \\
                    && Induktivaufnehmer\\
                    \hline
								
                    Wirbelstrom & Änderung des Wirbelstroms & Induktive Initiatoren\\
                    \hline
                    
                    Magnetfeld & Änderung des Magnetfelds & Hall-Generator\\
                    && Feldplatte\\
                    \hline

                    Optoelektisch & Opto-elektrische Umsetzung & Fototransistor\\
                    && Fotodiode\\
                    

                \end{tabularx}\\
                \caption{Arten von Sensoren}
                \label{fig:TS09}
            \end{table}

                
                Desweiteren wird zwischen aktiven und passiven Sensoren unterschieden. 
                
                Ein aktiver Sensor ist selbst ein Spannungserzeuger und benötigt keine weitere elektrische Hilfsenergie, um betrieben zu werden. Beispiele dazu wären ein Thermoelement, Licht- oder Drucksensor.\\
                
                Passive Sensoren hingegen benötigen eine aktive Spannungsversorgung, auch Sekundärelektronik genannt, welche es erlaubt, die Messung in elektrische Signale, also in Primärelektronik, umzuwandeln. Beispiele dazu wären eine Wägezelle, Dehnungsmessstreifen, Magnetfeldsensoren.		
                            
                                        
\subsection{Klassische Sensoren im Automobil} 
        
    Nach dieser allgemeinen Grobeinteilung von Sensoren, soll nun im Weiteren spezieller auf die Sensortechnik in Kraftfahrzeugen eingegangen werden.

    Der Grund für die immer umfangereichere Sensortechnik in Automobilen liegt in der Veränderung des Automobils im luafe der Zeit, insbesondere im Hinblick auf Fahrunterstützung und Fahrhilfssysteme.
    
    Die Sensoren dienen hierbei der Eingabe,  bzw. dem Auslesen eines tatsächlichen Wertes (Ist-Wert). Diese werden einer ECU über ein Bussystem übermittelt, welches einen Abgleich des Ist-Wertes mit einem vorgegebenen Soll-Wert vornimmt und dementsprechend das Modul nachregeln und steuern kann. \\ 
    
    Da Sensoren diversen äußeren physikalischen und/ oder chemischen Einwirkungen ausgesetzt sind, müssen sie diesen entsprechend gerecht werden. Aus diesem Grund gibt es verschiedene Anfertigungsformen wie wasserdicht, dreck- und staubgeschützt. Diese werden in dem nächsten Kapitel ebenfalls erörtert.\\  
            
        Ein Ausfall des Sensors kann durch folgende Ursachen hervorgerufen werden:

        \begin{itemize}
            \item Kurzschlussbegin
            \item Leitungsunterbrechung
            \item Verschmutzung
            \item Mechanische Beschädigung
            \item nicht korrektes Anbringen des Sensors
            \item Sensor defekt
        \end{itemize}	
        
        Nach einem Ausfall kann man durch gezielte Fehlersuche der Fehler analysiert werden:

        \begin{itemize}
            \item Anschlüsse prüfen
            \item Auslesen des Fehlerspeichers
            \item Allgemeine optische Prüfung
            \item Säubern der Sensoren
            \item Messungen mit einem Messinstrument vornehmen wie:
            \begin{itemize}
                \item Oszilloskop 
                \item Voltmeter
                \item Amperemeter
                \item Ohmmeter	
            \end{itemize}
            
        \end{itemize}

        \subsubsection{Temperatursensor NTC}
			Bei einem Temperatursensor handelt es sich um einen nicht-mechanischen, aktiven	Sensor, welcher die Temperatur erfasst.\\
            Die Sensoren sind auf einen Messbereich zwischen -40$^\circ$C und +200$^\circ$C skaliert.
            Für Temperaturen, die über diesen Bereich herausgehen, werden spezielle Hochtemperatursensoren (HTS) verwendet.

            \begin{figure}
				\centering
                \includegraphics[width=\textwidth] {aufbau_ntc.png}
                \caption[www.kfztech.de/kfztechnik/elo/sensoren/ntc.htm]{Aufbau NTC}
                \label{fig:TS02}
			\end{figure}

            Abbildung \ref{fig:TS02} zeigt den Aufbau eines Temperatursensors.

            Hierbei werden zwei unterschiedlich, elektrisch aufgeladene Metalle am Ende (der Messstelle) miteinander verbunden. Hier wird die Temperatur von dem zu messenden Element gemessen.\\
            Die Temperaturdifferenz wird an den zwei offenen Enden mit Hilfe des Seebeck-Effekts in elektrische Spannung umgewandelt. 
            Dann kann die Differenz der beiden Leiter verglichen werden, wobei einer der Leiter als Referenz benutzt wird (grauer Leiter).\cite{TS_temp} 
					
			NTC Sensoren werden beispielsweise in folgenden Komponenten verwendet:
			
			\begin{itemize}
				\item Katalysator
				\item Öltemperatur
				\item Klimaanlagen
				\item Kühlwassertemperatur 	
			\end{itemize}
			
            HTS hingegen werden für
            
			\begin{itemize}
				\item AGR
				\item Turbolader
				\item Rußfilter 
            \end{itemize}
            
            eingesetzt.

            \subsubsection{Induktive Sensoren}
		
            Induktive Sensoren arbeiten nach dem Induktionsgesetz.
            
			Hierbei erleidet der Sensor keinerlei Verschleiß, da kein direkter Kontakt zu dem zu messenden Objekt existiert.\\
			Für ein Verständnis, wo diese Sensoren welche Aufgaben im KFZ- Bereich erfüllt, reicht es nur den groben Aufbau zu kennen.\\
			Eine Spule wird von Strom durchflossen, welche daraufhin ein Magnetfeld erzeugt.\\	

			In Abbildung \ref{fig:TS03} ist der Sensor über einer Art Zahnrad, welches man Impulsrad nennt, angebracht.\\
            
            "Der magnetische Fluss durch die Spule hängt davon ab, ob dem Sensor eine Lücke oder ein Zahn gegenübersteht. Ein Zahn bündelt den Streufluss des Magneten, eine Lücke dagegen schwächt den Magnetfluss." \cite{TS_ind_funkt}  
			Die Anzahl der Änderungen/ Impulse in einer bestimmten Zeiteinheit ist ein Maß für die Drehzahl des zu messenden Objekts. Darüber hinaus kann auch die exakte Position des Moduls erkannt werden.\\					
            
                \begin{figure}
                    \centering
			    	\includegraphics{Induktiv_mit_legende.jpg}
                    \caption[www.kfztech.de/kfztechnik/elo/sensoren/induktivgeber.htm]{Induktiver Sensor}
                    \label{fig:TS03}
                \end{figure}

			Induktive Sensoren werden beispielsweise in folgenden Modulen verwendet:
			
			\begin{itemize}
				\item Drehzahlerfassung wie
					\begin{itemize}
						\item Kurbelwellenstellung
						\item Getriebe
					\end{itemize}	
				\item Drosselklappenstellung
				\item Lenkwinkel z.b. ESP
				\item Fahrpedalgeber
				\item Bremspedal
			\end{itemize}
		
			Allerdings kann der Ausfall eines Sensors zu erheblichen Schäden und sicherheitstechnischen Gefährdungen führen:
			\begin{itemize}
				\item Motor kann aussetzen oder gar stillstehen
				\item es wird ein Fehlercode abgespeichert
            \end{itemize}
           
            \subsubsection{Ölsensor}
					
            Dieser Sensor wird verwendet, um Informationen über den Zustand und den Füllstand des Öls zu erhalten. So kann angezeigt werden, ob und wann ein Ölwechsel notwendig ist. 
            Das Einbringen des Ölsensors spart also Geld, schont die Umwelt und gibt Rückschlüsse auf den Zustand des Motors und es können Motorschäden verhindern.\cite{TS_oel}
            
            \subsubsection{Lenkdrehmomentsensor}

            ``Um die Funktion des Servomoduls umsetzen zu können, benötigt das Steuergerät exakte Informationen über die vom Fahrer eingegebene Lenkbefehle. Um diese Eingabe richtig erfassen zu können, wurde ein Sensor konstruiert, welcher [...] die erforderlichen Daten wie Drehwinkel, Drehrichtung und Drehmoment elektronisch erfassen.[...] ``  \cite{TS_dreh} kann und an das Steuergerät sendet.\\
            
            Die Funktionsweise eines Drehmomentsensors kann auf zwei Arten realisiert werden:\\
            
            \begin{enumerate}
                \item \textbf{Magnetoresistives Prinzip}:
                
                         Mittels Eingangswelle, Torsionsstab/ Drehstab (eine stabförmige Feder, welche sich in axialer Richtung drehen lässt), Antriebswelle und einem magnetoresistivem Element. \\
                         Damit Leitungen zur Spannungsversorgung und Signalübertragung nicht beschädigt werden, sind diese in einer Vermantelung, welche auch als Wickelkassette bezeichnet wird.\\
                         
                         Durch das Einlenken des Fahrers wird der Torisonsstab ebenfalls verdreht. ``Diese Verdrehung ist ein Maß für das Lenkmoment.`` \cite{TS_dreh}

                         Durch gezieltes Aufbringen von Widerständen können die Drehbewegungen registriert werden. Durch das Drehen, also spannen oder stauchen der Feder ( Fahrer lenkt links bzw rechts ein) verändert sich das Magnetfeld, welches wiederum den elektrischen Widerstand verändert und somit die über dem Widerstand anliegende Spannung. Diese Spannungssignale werden über die Signalleitungen an das Steuergerät übersendet, welches aus den Informationen die aufzubringende Unterstützungsmomente berechnen kann.				 

                \item \textbf{Optisches Prinzip}:\\

                        Vor und hinter dem Torsionsstab ist jeweils eine Scheibe angebracht, welche eine bestimmte Codierung mittels eingelassenen Löchern hat (siehe Abbildung \ref{fig:TS05}).\\
                        Zur Bestimmung der Einlenkgeschwindigkeit wird axial und parallel zu dem Torsionsstab eine Leuchtdiode (über der ersten Scheibe) und eine Fotodiode (unter der zweiten Scheibe) angebracht. \\
                        Die Leuchtdiode sendet gebündeltes Licht aus, welches auf der gegenüberliegenden Seite von der Fotodiode erkannt wird. Bei einfallendem Licht verändert sich der durchfließende elektrische Strom durch die Fotodiode.\\
                        Dreht sich also die Scheibe kommt es zu einem Wechsel der Stromstärke. Diese Impulse werden an das angebundene Steuergerät gesendet, welches aus diesen empfangenden Informationen die Drehgeschwindigkeit berechnen kann.\\
                        
                        \begin{figure}
                            \centering
                            \includegraphics{photoelektrisch.png}
                             \caption[www.kfztech.de/kfztechnik/fahrwerk/lenkung/photoelektrisch.jpg]{Photooptisches Prinzip}
                             \label{fig:TS05}
                        \end{figure}	
            \end{enumerate}

            \subsubsection{Regensensor}

                Dieser nicht-mechanische Sensor ``[...] registriert Wassertropfen auf der Windschutzscheibe durch opto-elektrisches Verfahren`` \cite{TS_regen}
                
                Der Regensensor befindet sich innerhalb des Wischbereichs des Scheibenwischers.                
				In dem Sensor befindet sich eine Leuchtdiode und eine Fotodiode, welche in einem bestimmten Abstand voneinander angebracht sind. Die Leuchtdiode sendet ein Infrarotlicht aus. Dieses Lichtbündel wird bei trockener Windschutzscheibe an der äußeren Scheibe reflektiert und nahezu mit voller Lichtstärke von dem Sensor aufgenommen. In der Physik spricht man hier von einer Totalreflexion.\\
				Befinden sich nun Wassertropfen in dem Bereich des Sensors auf der Frontscheibe wird das ausgesendete Lichtbündel nicht komplett reflektiert, sondern ein Teil des Lichtes wird gebrochen und durch den Tropfen gestreut. Das Resultat daraus ist, dass der ausgesendete Lichtstrahl nur noch mit einem Bruchteil der ursprünglichen Stärke den Sensor erreicht.\\
				Aus diesen Daten errechnet die dort darin befindliche Elektronik die Stärke des Niederschlages, gibt diese an ein Steuergerät weiter, welches wiederum die Scheibenwischer ansteuert. Somit bleibt die Oberfläche des Sensors immer tropfenfrei und es wird ein optimales Messergebnis erreicht. (Fig. 8)
				
				\begin{figure}
					\centering
					\includegraphics{regensensor2.jpg}
                    \caption[archiv.langzeittest.de/volvo-s40/intern/grafik/cb-regensensor-prinzip.jpg] {Prinzip eines Regensensors}
                    \label{fig:TS06}
                \end{figure}
                
            \subsubsection{Seitenwandtorsionssensor}

				Elektronische Regelsysteme wie ABS oder ESP benötigen sämtliche Informationen über das Zusammenspiel zwischen Fahrzeug und dem Fahruntergrund und den daraus entstehenden Kräften. Um die einzelnen Sekundärgrößen wie Motorleistung, Bremsdruck, Radgeschwindigkeit und Fahrzeugbeschleunigung zur Berechnung nicht mehr benutzen zu müssen, wurde von Continental ein Sensorsystem entwickelt, welches das Rad als Sensor fungieren lässt, das sogenannte SWT-System. 
				
				\begin{figure}
					\centering
					\includegraphics{swt1.png}
                    \caption[www.kfztech.de/kfztechnik/sicherheit/swt/swt1.gif]{Reifen als Sensor}
                    \label{fig:TS07}
				\end{figure}
			
                Der Reifen besteht aus Magnetgummi und Magnetfeldsensoren, sowie einem Signalaufbereitungssystem und einer zentralen Recheneinheit.
					
                Die Messung findet mittels der Verformung des Reifens statt, die entsteht wenn der Reifen bei Kurvenfahrten durch die einwirkenden Querkräfte temporär aus der Form gebracht wird. 
                
                Hierfür werden zwei Sensoren am Fahrwerk angebracht, wobei einer der beiden auf der Höhe der Felge ist und der andere nahe am Scheitelpunkt des Reifens angebracht wird. Darüber hinaus ist die Reifenseitenwand magnetisch, um ein Messergebnis erzielen zu können.\\
                In der obrigen Abbildung erkennen wir ein Muster auf dem Reifen. Dies ist darauf zurückzuführen, dass ein Magnetpulver in die Reifenseitenwand eingemischt wird. Dieses Gemisch wird über den gesamten Umfang des Reifens gestrichen und somit erhält man eine alterniernde Nord- und Südpole, was durch die Streifen visualisiert wird.
                
                Solange auf den Reifen keine Längskräfte wirken, ``[]...] erfolgt der Wechsel zwischen den Magnetpolen an beiden Sensoren gleichzeitig, die Zeitdifferenz zwischen den Signalen beider Sensoren ist Null`` \cite{TS_swt} \\	
                Beschleunigt oder verzögert der Fahrer das Fahrzeug, überschreiten die Grenzen der Magnete zu unterschiedlichen Zeiten die Sensoren, somit ist eine Zeitdifferenz zu messen. 
                Diese gemessene Zeitdifferenz wird an ein eingebautes Steuergerät gesendet, welche die Fahrassistenzsysteme wie ASR und ABS anspricht und dementsprechend der derzeitigen Fahrsituation anpasst.\\
                
				Während einer Kurvenfahrt kann der Abstand zwischen der Reifenseitenwand und dem Sensor gemessenwerden, da sich dieser mechanisch gesehengesehen verkürzt bzw. vergrößert wird und sich somit die Stärke des Magnetfelds ändert.\\
                Mit Hilfe dieser gemessenen Werten können Hilfssysteme wie ASB und ESP entsprechend angesteuert werden. Dies sorgt für ein sichereres Fahren da die Ansteuerung optimiert werden kann. 
                
                In der Praxis bedeutet dies, dass der Fahrer einen kürzeren Bremsweg und eine bessere Kontrolle über das Fahrzeug auf kurvenreichen und schweren Strecken hat.

                \subsubsection{Reifensensor}

                Ein in den Reifenprofil eingebetteter Chip überträgt hochfrequent die Signale an eine im Radkasten befindliche Antenne. 
                Ändert sich der Zustand der Straße verändert der Sensor das Signal. Dies geschieht pro Sekunde mehrere Male. 
                Darüberhinaus kontrolliert der Sensor permanent den Reifendruck. Mit diesen Informationen kann nicht nur die Lebensdauer des Reifens, sondern auch die Sicherheit erhöht werden, 
                da sich gezielt ABS und ESP einschalten.\\ 
			
                \subsubsection{Reifendrucküberwachungssystem}
                Reifendrucküberwachungssysteme (RDKS) werden eingesetzt, um die Lebensdauer des Reifens zu erhöhen. Seit 1. November 2014 sind die Reifen- und Autohersteller verpflichtet, ein RKDS zu implementieren.
                Hierbei wird jeder Reifen mit einem Sensor ausgestattet, damit der Fahrer Luftverlust oder gar einen Plattfuß bemerken kann.
                
                Prinzipiell gibt es RDKS in aktiver und passiver Variante. In Tabelle \ref{fig:TS08} werden diese in Funktionsweise mit ihren jeweiligen
                Vor- und Nachteilen gegenübergestellt.

                \begin{table}
                    \begin{tabularx}{\textwidth} {l|l}
                        
                        
                        \multicolumn{2}{c}{\textbf{Passives Reifendrucküberwachungssystem}}\\
                        \hline
                        \textbf{Erklärung:} & Platter Reifen hat einen kleineren Abrollumfang \\ & und dreht sich schneller.\\ & ABS Sensoren messen dies und das Steuergerät erkennt dies\\
                        \hline
                        \textbf{Vorteil:} & kostengünstig, in Kombination mit Runflat Tires eine \\ & optimale Lösung, da nur eine geänderte Software \\ & und Kontrolle erforderlich sind.\\
                        \hline
                        \textbf{Nachteil:} & schleichender Luftverlust der Räder an einer Achse \\ &  wird nicht erkannt. Nur Differenzen größer 0.5 Bar werden \\ & während der Fahrt erkannt. Höherer Spritverbrauch durch \\ & zu niedrigen Luftdruck.\\
                        \multicolumn{2}{c}{\textbf{Aktives Reifendrucküberwachungssystem}}\\
                        \hline
                        \textbf{Erklärung:} & Jeder Reifen besitzt eigenen Sensor und sendet Informationen \\ & über Druck und Temperatur per Funk an Steuergerät.\\
                        \hline
                        \textbf{Vorteil:} & Exakte Messung, ab einer Differenz von 0.2 Bar wird ein \\ & Alarm ausgelöst. Reserverad wird mit überwacht.\\
                        \hline
                        \textbf{Nachteil:}  & Teuer für Reifenmontage, da Reifen kodiert werden müssen\\ & und der Reifenwechsel aufwendig wird.\\

                    \end{tabularx}
                    \caption{Aktive und Passive Reifendrucküberwachungssysteme \cite{TS_rdks}}
                    \label{fig:TS08}

                \end{table}

                
                Die Reifenelektronik sitzt auf der Innenseite des Reifen und misst in kurzen definierten Zeitabständen Reifendruck und Temperatur. 
                Jeder Sensor ist mit einer eigenen ID- Nummer ausgestattet. Per Funk werden an die eigene Empfangsstation Datenpakete geschickt, welche die ID, sowie Druck und Temperatur des Reifens beinhalten.

                Dieses Empfangsgerät sendet die empfangenen Daten kabelgebunden weiter an das Steuergerät. Dieses wertet die Daten aus und sendet bei Bedarf, sprich Unter- oder Überschreitung der Sollwerte, eine Meldung an die Kontrollanzeige.
				
                \subsubsection{Hallsensor}

				Hallsensoren werden eingesetzt in: 
				\begin{itemize}
					\item Zündanlage
					\item Getriebeausgabedrehzahl
					\item Radlöseerkennung/- warnung (Audi)
					\item aktiver Drehzahlsensor in ABS- Bremsanlagen
					\item Nockenwelle (Koordination von Eispritzbeginnberechnung oder Pumpe-Düse, sowie Klopfregelung)
				\end{itemize}
				
                Am Beispiel des Nockenwellensensors wird im Folgenden die Funktion des Sensors beschrieben.
                
                Die Nockenwelle bringt einen aus ferromagnetischem Material angefertigten Rotor zum Drehen. Zwischen Rotor und einem Dauermagnet befindet sich der Sensor.
                Durch den Dauermagneten wird ein Magnetfeld erzeugt, welches den Sensor senkrecht durchfließt. Passiert ein metallischer Gegenstand, beispielsweise ein Zahn der Nockenwelle, verändert dieser das Magnetfeld, welches den Sensor durchfließt.
				Elektronen werden senkrecht zum Magnetfeld stärker abgelenkt, wobei eine Hall-Spannung von mehreren Millivolt erzeugt. Eine integrierte Auswerteelektronik bereitet das Signal auf und leitet es an in Form eines Rechtecksignals an das Steuergerät geleitet. \cite{TS_hall}

				\begin{figure}
					\centering
					\includegraphics[width=\textwidth]{hall.png}
					\caption[www.kfztech.de/kfztechnik/elo/sensoren/hallsensor.htm]{Aufbau eines Hall-Sensors}
                \end{figure}
                
                \subsubsection{Rad- Drehzahlsensor}

				Durch drei entsprechend angeordnete Sensoren kann die Drehrichtung des Rades erkannt werden. Ein entsprechend angeordneter Magnet (Abbildung \ref{fig:TS10}) ersetzt hierbei die Funktion der mechanischen Zahnräder.\\
                \textbf{Aktive Sensoren}: Dieser wird bereits mit Spannung versorgt und erzeugt aus dem wechselnden Magnetfeld ein Rechtecksignal und sendet diese unverändert an ein Steuergerät.
                
				Dies ermöglicht ebenso eine Geschwindigkeitsmessung von 0.1km/h. Diese Werte kann man zum Beispiel für Einparksysteme oder Navigationssysteme benutzen.\cite{TS_drehzahl_sensor}
				(Abbildung \ref{fig:TS11})

                \begin{figure}[h!]
                    \begin{minipage}[b]{.4\linewidth} % [b] => Ausrichtung an \caption
                       \includegraphics[width=\linewidth]{radsensor.png}
                       \caption{Aufbau}
                       \label{fig:TS10}
                    \end{minipage}
                    \hspace{.1\linewidth}% Abstand zwischen Bilder
                    \begin{minipage}[b]{.4\linewidth} % [b] => Ausrichtung an \caption
                       \includegraphics[width=\linewidth]{signalverlauf_hall.png}
                       \caption[www.kfztech.de/kfztechnik/elo/sensoren/drehzahlsensor.htm]{Signalverlauf}
                       \label{fig:TS11}
                    \end{minipage}
                 \end{figure}

                 \subsubsection{Airbagsensor}

                 Für das Auslösen des Airbags werden mehrere Frontsensoren verwendet, welche an diversen Stellen im Auto angebracht sind. Die Beschleunigungssensoren messen einwirkende Kräfte und geben ein Signal an das Steuergerät.
                 Damit der Airbag nicht fälschlicherweise geöffnet wird, muss ein Sicherheitssensor innerhalb des Steuergerätes dies bestätigen. 
                 
                 ``Durch das genau Erfassen der Unfallschwere wird die Auslösung der Airbags und der Gurtstraffer aktiviert.`` \cite{TS_airbag}\\
                 Faktoren sind: 
                 
                 \begin{itemize}
                     \item Aufprallstärke
                     \item Sicherheitsgurte angelegt
                     \item Sitzposition des Fahrers und Beifahrers
                 \end{itemize}
             
             
             \subsubsection{Positionssensoren}

                 Diese Sensoren sind an diversen relevanten Stellen des Automobils befestigt. Sie geben Aufschluss über Ggenstände in unmittelbarer Nähe des Fahrzeug.

                 Um diese erkennen zu können wird ein Ultraschallsensor eingesetzt, welcher ein Signal aussendet. Falls ein Gegenstand in der Nähe ist, reflektiert dieser die Schallwellen und ein Ultraschallempfänger empfängt diese. 
                 Durch die Zeitdifferenz des ausgesendeten und des erhaltenen Signals kann die Entfernung errechnet werden. Kommt das Objekt näher, so verkürzt sich der Abstand und somit auch die Zeit des Eintreffen des Schalls.

                 Der Sensor sendet zu einen die Zeit, wann dieser die Ultraschallwellen gesendet hat sowie die reflektierten Signale. Aus dieser Differenz kann  der Abstand zwischen Auto und Gegenstand errechnet werden. 
                 Nähert sich das Auto einem Objekt, so verkürzt sich die Zeit zwischen den austretenden und empfangenen Schallwellen. Bei dem kleinsten zulässigen Abstand meldet das Steuergerät dies der Kontrolleinheit und der Fahrer wird informiert.

                 Falls sich kein Objekt in der Nähe befindet wird auch kein Signal reflektiert und somit erkennt das Steuergerät, dass keine Kollision stattfinden kann.

                \subsection{Smarte Sensoren} 
                 Unter smarten Sensoren versteht man Sensoren, die über eine integrierte Recheneinheit und -logik verfügen.
                 
                 Dadurch können diese Sensoren neben dem reinen Messen die eingelesenen Daten direkt verarbeiten und in diesem Zustand dem Steuergerät überreichen. 
                 
                 Dies hat den Vorteil, dass das Steuergerät keine überflüssigen Rechenschritte abarbeiten muss und somit mehr Rechenleistung für die Reaktion auf besondere Ereignisse hat.

                 Smarte Sensoren können beispielsweise bei Feldbussystemen wie LIN, CAN, Flexray eingesetzt werden.		
             
             
             
             \subsection{Zukunftsvisionen} 
                  Durch das Weiterentwickeln der Sensortechnik können beispielsweise Unfälle vermieden werden, da der Sensor eine deutlich schnellere Reaktionszeit aufweist als ein Mensch.
                  Ein Beispiel wäre, das bestehende Positionssensorsystem durch eine neue Art der Messung zu ersetzen. Hierbei werden die Ultraschallsensoren 
                  durch sogenannte Radarkameras ersetzt, welche einen deutlich weiteren Radius abdecken können, als die bisher herkömmlichen Sensoren.(Fig. 18)
                  
                  \begin{figure}
                      \includegraphics[width=\textwidth] {radarsensor.png}
                      \caption[www.leifiphysik.de/akustik/schallgeschwindigkeit/ausblick/ultraschall-beim-auto]{Unterschied: Video-, Infrarot- und Radarkamera}
                      \label{fig:TS12}
                  \end{figure}
     
                     In Abbildung \ref{fig:TS12} kann man deutlich den Unterschied in der Reichweite der verschiedenen Sensoren sehen.\\
                     
                     Neben diesem Effekt kann durch neue entwickelten Sensoren beispielsweise der Spritverbrauch und somit den Ausstoß an $CO_2$ verringert werden.\\
                     Auch im Hinblick auf Elektromobilität wird es neue Sensoren geben, welche zur Überwachung der Batterie eingesetzt werden.

               
                
          



        
    

    
%\section{ECU / Steuergeräte}
    \subsection{Introduction}
    The ECU is an electronic control unit and thi
\section{Fahrerassistenzsysteme}
    \subsection{Einführung}
    Fahrerassistenzsysteme sind Systeme die elektrisch den Autofahrer in bestimmten
    Situationen helfen und unterstützen. Je nach Marke, Modell und Land können
    unterschiedliche Kombinationen von Fahrerassistenzsysteme hinzugebucht werden
    oder sind als Standardequipment bereits im Fahrzeug eingebaut. Solche Assistenten
    werden für die Sicherheit und den Komfort in die Automobile eingebaut. Für die
    einzelnen Systeme werden verschiedene Geräte wie Sensoren, Radar, Video oder
    Ultraschall verwendet, um ausreichend Informationen für die einzelnen Assistenten
    zu bekommen. Wenn nun eine riskante Situation aus der Sicht der Systeme entsteht,
    wird diese dem Fahrer durch visuelle und akustische Signale mitgeteilt. Solche
    Fahrerassistenten ersetzen bei manchen Stellen komplett den Fahrer und dies ist
    der Weg zum autonomen Fahren.
    ~\cite{assistenzsysteme.PB1} ~\cite{assistenzsysteme.PB2} ~\cite{assistenzsysteme.PB3}
    ~\cite{assistenzsysteme.PB4}
    \subsection{Fahrerassistenzsysteme}

        \subsubsection{Antiblockiersystem (ABS)}
        ABS (Antiblockiersystem) verhindert, dass bei einer Vollbremsung die Räder
        blockieren und der Fahrer die Kontrolle über das Fahrzeug verliert.
        Um die Kontrolle nicht zu verlieren, wird in kurzen Abständen immer wieder
        der Bremsvorgang gelöst.
        ~\cite{assistenzsysteme.PB2} ~\cite{antiblockiersys.PB1}

        \subsubsection{Elektronisches Stabilitätsprogramm}
        (Electronic stability control) / \textbf{ESP} (Elektronisches Stabilitätsprogramm)
        ist ein Sicherheitssystem zur Spur- und Stabilitätskontrolle des Fahrzeugs.
        Es soll als Unterstützung bei Ausweichmanövern oder beim Verlust der Kontrolle über das Fahrzeug dienen. 
        Zu diesem Zweck wird das Drehmoment des Motors reduziert, sollte dies nicht ausreichen leitet das System einen aktiven
        Bremsvorgang ein.
        ~\cite{assistenzsysteme.PB2} ~\cite{ESP.PB1}
        
        \subsubsection{Antriebsschlupfregelung (ASR)}
        Ein Assistent für die Unterstützung beim Anfahren oder Beschleunigen. Der
        Assistent sorgt für eine gleichmäßge Verteilung des Drehmoments auf die Räder.
        Das Endziel soll verhindern, dass die Räder durchdrehen. Dadurch soll eine 
        bessere Fahrstabilität und Fahrtraktion entstehen. Unter anderem soll auch 
        die Fahrzeugführung und die Lenkung verbessert werden.
        ~\cite{assistenzsysteme.PB2} ~\cite{ASR.PB1} ~\cite{ASR.PB2} 

        \subsubsection{Bremsassistent (BAS)}
        Beim Bremsen sorgt dieser Assistent für eine Verstärkung des Bremsdruck, sodass
        eine Vollbremsung bei einer Notfallsituation entsteht. Der Assistent soll Unfälle 
        verhindern oder zumindest abschwächen, indem er den Fahrer bei dem Bremsvorgang 
        unterstützt.
        ~\cite{assistenzsysteme.PB2} ~\cite{bremsassi.PB1} ~\cite{bremsassi.PB2}

        \subsubsection{Berganfahrhilfe}
        Aktiviert eine automatische Handbremse, sodass beim Anfahren am Berg kein Rückwärtsrollen
        entsteht. Der Assistent soll das hektische Bremspedal lösen und Gaspedal treten 
        verhindern.
        ~\cite{berganfahr.PB1} ~\cite{berganfahr.PB2}  ~\cite{assistenzsysteme.PB2}
        
        \subsubsection{Bergabfahrhilfe (HDC)}
        Beim Berg Herabfahren regelt dieser Assistent die Geschwindigkeit bei steilen Hängen. 
        Ohne diesen Assistenten regelt der Motor die Geschwindigkeit, wenn das Gaspedal nicht 
        betätigt wird. Bei höhrerem Gewicht des Automobils ist es nicht mehr möglich, dass 
        der Motor diese Arbeit selbständig erledigt, sodass der Motor durch zusätzliches Bremsen 
        des HDC unterstützt wird.
        ~\cite{assistenzsysteme.PB2} ~\cite{bergabfahr.PB1} 

        \subsubsection{Abstandsregeltempomat (ACC, Adaptiv Cruise Control)}
        Der Assistent sorgt für die passende Geschwindigkeit, sodass der richtige Abstand
        zu dem vorausfahrenden Fahrzeug eingehalten wird. Wenn man zu nahe auf das vordere
        Fahrzeug auffährt, wird abgebremst oder bei zu großem Abstand beschleunigt. Der Abstand
        den das Fahrzeug einhalten soll kann in dem Assistenten eingestellt werden, manche Fahrzeuge
        haben die Möglichkeiten von kurz, mittel und groß. Wird von dem Assistent kein vorausfahrendes
        Fahrzeug erkannt, so arbeitet der Assistent als Geschwindigkeitsregeler für den Fahrer. Der
        neue Assistent soll zusätzlich das Abbremsen bis zum Stillstand und Stop\&Go beherrschen.
        ~\cite{assistenzsysteme.PB2} ~\cite{Audi.PB1}

        \subsubsection{Automatisches Notbremssystem (AEBS)}
        Dieser Fahrerassistent soll selbständig mögliche Zusammenstöße erkennen. Wenn sich solch
        eine Möglichkeit bietet, soll der Assistent das Fahrzeug abbremsen lassen und somit einen
        Zusammenstoß verhindern. Dieser Assistent geht so weit, bis das Fahrzeug zum Stillstand 
        gebracht ist.
        ~\cite{notbremsassi.PB1} ~\cite{assistenzsysteme.PB1}  ~\cite{assistenzsysteme.PB2}
        ~\cite{notbremsassi.PB2}
        
        \subsubsection{Spurhalteassistent (LKA, Lane Keeping Assistent)}
        Ein System, das den Fahrer unterstützt eine optimale Spur auf der Fahrbahn beizubehalten.
        Das Fahrzeug soll die Position im Bezug zu der Spur- und Straßenbegrenzung halten. Die
        Spur wird durch leichte Lenkeingriffe bei dem Fahrzeug beibehalten. Der Fahrer kann aber
        jederzeit selber in das Fahrverhalten eingreifen.
        ~\cite{spurhalte.PB1} ~\cite{assistenzsysteme.PB1} ~\cite{spurhalte.PB2}  ~\cite{assistenzsysteme.PB2}

        \subsubsection{Spurverlassenswarner (LDW, Lane Departure Warning)}
        Eine Warnfunktion die dem Fahrer hilft und ihn warnt, wenn das Automobil die Fahrspur
        verlassen sollte. Voraussetzung für eine Warnung ist, dass der Fahrer keinen Richtungsblinker
        gesetzt hat und ein Fahrspurwechsel beabsichtigt. Der Blinker wird als Signal gesehen, ob das 
        Fahrzeug die Spur aktiv verlassen will oder ob es ein Abkommen von der Spur ist. Die 
        Warnung kann auf verschiedene Arten auftreten: Lenkeingriff, Lenkradvibration oder visuelles
        Signal.
        ~\cite{assistenzsysteme.PB2} ~\cite{LDW.PB1}

        \subsubsection{Überholassistent}
        Ein Assistent der dem Fahrer hilft ein Überholmanöver durchzuführen oder sogar ein
        komplettes Überholmanöver selber durchführt. Beim Überholmanöver ist der Assistent 
        behilflich, indem er dem Fahrer Informationen über den toten Winkel gibt. Dadurch 
        wird dem Fahrer geholfen, ob beim Verlassen der Spur von hinten ein Fahrzeug kommt 
        oder nicht. Der Überholassistent wird auch Spurwechselassistent genannt.
        ~\cite{ueberholassi.PB1} ~\cite{spurwechsel.PB1} ~\cite{assistenzsysteme.PB1} 
        ~\cite{assistenzsysteme.PB2}
        
        \subsubsection{Fernlichtassistenten}
        Dieser Assistent bietet einen großen Sicherheitsgewinn. Er sorgt für eine bessere
        Ausleuchtung der Straße und kann bei neuen Modellen dynamisch eingesetzt werden.
        Diese Funktion bietet dem Fahrer, dass dieser sich nicht auf das Einschalten und
        Abblenden des Fernlichts konzentrieren muss. Dies wird von dem Assistent übernommen,
        dieser erkennt ob Gegenverkehr kommt und blendet ab und danach schaltet er das Fernlicht
        wieder ein.
        ~\cite{assistenzsysteme.PB2} ~\cite{Audi.PB1}

        \subsubsection{Speed Limiter}
        Der Speed Limiter ist eine Funktion, die eine Obergrenze für die Geschwindigkeit festlegt.
        Man kann bei dem Speed Limiter eine Geschwindigkeit einstellen 
        und dann ist es egal wie stark man auf das Gaspedal drückt. Die Geschwindigkeit wird nicht 
        über die eingestellte Geschwindigkeit gehen. Es soll helfen die Geschwindigkeit besser 
        einzuhalten und kann auch mit der Verkehrszeichenerkennung kombiniert werden.
        ~\cite{assistenzsysteme.PB2} ~\cite{assistenzsysteme.PB2} 

        \subsubsection{Intelligent Speed Adaptation (ISA)}
        Das System soll den Fahrer unterstützen, indem es Rückmeldungen bei überhöhter Geschwindigkeit
        zurückgibt. Das System soll die Verkehrsbedingung und Straßenverhältnisse beurteilen und
        anhand diesen Informationen geeignete Rückmeldungen geben. Unter anderem wir die Position des 
        Fahrzeugs über das GPS festgestellt und somit dann auch das Geschwindigkeitslimit ermittelt.
        ~\cite{ISA.PB1}  ~\cite{speedlimiter.PB1} ~\cite{ISA.PB2}

        \subsubsection{Erkennung und Notbremsung beim Rückwärtsfahren (Reversing detection)}
        Ein Assistent der Informationen über Objekte hinter dem Fahrzeug übermittelt. Diese Funktion
        wird beim Rückwärtsfahren benötigt, um keinen Zusammenstoß zu verursachen. Dem Fahrer wird 
        über akustische Signale mitgeteilt, ob sich hinter dem Auto ein Objekt befindet oder nicht.
        Im Notfall kann das System eingreifen und eine Notbremsung durchführen.
        ~\cite{assistenzsysteme.PB2}  ~\cite{reversedetection.PB1}

        \subsubsection{Abbiegeassistent}
        Es sollen Fußgänger, Fahrradfahrer und andere Objekte, die sich neben dem Automobil befinden
        oder näher kommen erkennen und davor warnen. Somit soll ein Unfall zwischen den Verkehrsteilnehmern
        verhindert werden. Die Warnung die dieser Assistent dem Fahrer bietet ist ein optisches Signal,
        welches am Seitenspiegel aufleuchtet.
        ~\cite{assistenzsysteme.PB2} ~\cite{abbiegeassi.PB1} ~\cite{abbiegeassi.PB2}

        \subsubsection{Fahrermüdigkeitserkennung und -aufmerksamkeitsüberwachung}
        Ein System das die Aufmerksamkeit des Fahrers kontrolliert. Das System soll den Fahrer durch
        bestimmte Methoden analysieren, ob der Fahrer noch wach ist. Sollte dies nicht der Fall sein,
        wird der Fahrer durch ein Signal gewarnt. Eine Methode zur Analyse ist das Lenkverhalten, jedoch
        unterscheiden sich die Methoden unter den Herstellern.
        ~\cite{muedigkeitsassi.PB1} ~\cite{assistenzsysteme.PB1}  ~\cite{assistenzsysteme.PB2}
        ~\cite{muedigkeitsassi.PB2}
        
        \subsubsection{Stauassistent}
        Wenn ein Fahrzeug in einen Stau gelangt, so kann der Stauassistent aktiviert werden. Sollte dieser
        aktiv sein, kann der Fahrer die Beschleunigung und das Bremsen dem System überlassen. Es wird
        jedoch verlangt, dass der Fahrer jederzeit in der Lage ist einzugreifen und das Fahrzeug übernehmen
        kann.
        ~\cite{stauassi.PB1} ~\cite{stauassi.PB2}
        
        \subsubsection{Notfallassistent}
        Der neue Audi besitzt einen Notfallassistenten, dieser soll das Auto bei einem Notfall übernehmen 
        und zum Stillstand bringen. Der Assistent überwacht ob die Hände des Fahrers am Lenkrad sind und 
        ob über eine gewisse Zeit das Gas- und Bremspedal betätigt wurde. Ist dies nicht der Fall, so 
        wird die Situation als Notfall kategorisiert und es werden bestimmte Maßnahmen eingeleitet.
        Zuerst wird versucht den Fahrer durch verschiedene Reize anzusprechen, dass er wieder die Kontrolle 
        über das Fahrzeug übernehmen kann. Sind diese Versuche erfolglos, so werden Maßnahmen für den 
        Notfall eingeleitet. Diese Maßnahmen haben das Ziel das Fahrzeug sicher zum Stehen zu bringen, einen Zugang des 
        Notarztes zu gewährleisten und einen Notruf abzusetzen.
        ~\cite{Audi.PB1}
        
        \subsubsection{Parksystem}
        Einer der bekanntesten Assistenten ist der Parkassistent. Der Parkassistent kann optische und 
        akustische Signale zur Unterstützung beim Einparken bereitstellen. Die akustischen Signale werden
        durch die Ultraschallsensoren gesteuert, das heißt wenn man zu nah an ein Objekt fährt, wird das Signal
        lauter. Das optische Signal wird durch eingebaute Kameras an bestimmten Stellen geliefert.
        In der Zwischenzeit wird sogar zwischen aktiven Parkassistenten und passiven Parkassistenten 
        unterschieden, beim passiven muss das Lenken selber übernommen werden und beim aktiven wird 
        dies von dem Assistenten übernommen.
        ~\cite{parkassi.PB1} ~\cite{assistenzsysteme.PB1} ~\cite{parkassi.PB2}
        
        \subsubsection{Kamerabasierte Verkehrszeichenerkennung}
        Oftmals wird vom Fahrer ein Verkehrszeichen übersehen oder nicht wahrgenommen. Mit der Verkehrszeichenerkennung 
        übernimmt eine Kamera im Fahrzeug diese Aufgabe für den Fahrer und kann ihn somit unterstützen.
        Eine große Hilfe ist die Verkehrszeichenerkennung bei Geschwindigkeitsbegrenzungen, bei denen der 
        Assistent die Schilder erfasst und den Fahrer warnt, wenn dieser zu schnell fährt. Der Warnhinweis 
        wird auf dem Display angezeigt und je nach Assistent verschwindet er nach einer gewissen Zeit oder 
        wenn die richtige Geschwindigkeit erreicht wurde.
        ~\cite{assistenzsysteme.PB1} ~\cite{verkehrszeichenerk.PB1} ~\cite{verkehrszeichenerk.PB2}
        
        \subsubsection{Kreuzungsassistent}
        Dieser Assistent soll Kollisionen mit Querverkehr verhindern. Der Kreuzungsassistent soll den Fahrer 
        unterstützen Objekte, die durch Gegenstände versteckt sind, zu erkennen. In solchen Situationen findet 
        man sich meistens an Kreuzungen wieder. Der Assistent soll zudem noch die Unaufmerksamkeit des 
        Fahrers kompensieren, wenn sich dieser auf andere Verkehrsteilnehmer konzentriert. Das Arbeiten  
        dieses Assistenten geschieht durch Sensoren, die das Umfeld links und rechts neben dem Fahrzeug im 
        Auge behalten. Der Kreuzungsassistent sollte immer eingeschaltet sein, sodass er in kritischen Situation 
        reagieren kann. Warnungen kann der Assistent über Anzeigen im Display als Bild darestellen, Ertönen eines Warnsignals
        und Aufblinken eines Symbols dem Fahrer übermitteln. Im Notfall, wenn der Fahrer nicht 
        reagiert, wird durch den Assistenten ein Notbremseingriff betätigt.
        ~\cite{kreuzungsassi.PB1} ~\cite{kreuzungsassi.PB2}

%\section{Ausblick}

    Neue Entwicklungsziele und Konzepte im Bereich des teil- und vollautonomen Fahrens stellen die elektronischen Fahrzeugsysteme vor Herausforderungen, die mit den heute
    verwendeten Systemen noch nicht bewältigt werden können.\\
    Die selbstständige Durchführung der Fahraufgabe durch ein elektronisches System oder die aktive Unterstützung eines menschlichen Fahrzeugführers
    bedingt eine umfangreiche Erfassung und Auswertung sämtlicher Fahrzeug- und Umgebungsdaten.\\
    Zu diesem Zweck müssen die klassischen Sensorsysteme um neue Systeme erweitert werden, die zum Beispiel optische Umweltdaten über Kameras liefern, Umgebungsscans
    mittels Radar, Lidar oder Ultraschall durchführen und die in der Lage sind, diese Daten in Echtzeit auszuwerten und zu verarbeiten. ~\cite{BP06}\\

    Während insbesondere zu Beginn der Entwicklungs- und Integrationsphase die Fahrzeuge noch autarke Einheiten sind, die alle für die
    Fahraufgabe notwendigen Daten eigenständig erheben müssen ist das langfristige Hauptziel von Industrie und Rechtsgebern eine umfassende Vernetzung
    sämtlicher Entitäten, die am Verkehrsgeschehen aktiv oder passiv partizipieren zu einem intelligenten Transportation System (ITS).\\

    Ein solches ITS soll den Teilnehmern innovative Dienste anbieten, mit dem Ziel das Verkehrsgeschehen
    effizient zu verwalten und zu koordinieren, die Umwelteinwirkungen durch Kraftfahrzeuge zu senken und erhöhte Sicherheit für alle 
    aktiv und passiv am Verkehrgeschehen beteiligten zu bieten. ~\cite{BP04} \cite{BP11}\\

    Denkbare Szenarien sind zum Beispiel der aktive Austausch von Geschwindigkeits- und Richtungsdaten von Fahrzeugen in der näheren Umggebung, um unnötige Brems-
    und Beschleunigungsvorgänge zu vermeiden, sowie die Übertragung auslastungsspezifischer Geschwindigkeitsbegrenzungen von smarten Verkehrsschildern
    direkt an die Fahrzeuge. Auch eine kontaktlose Erfassung und Verrechnung von nutzungsabhängigen Straßengebühren wäre denkbar und ist bereits im Einsatz.\\

    Dies sind nur einige Beispiele der Möglichkeiten eines voll vernetzten Transportsystems, das umfangreiche Datenerfassung mit Ad-Hoc Kommunikationsmöglichkeiten und
    modernen Prognose- und Analysealgorithmen vereint.

    Das Konzept eines solchen ITS bedingt jedoch, dass alle Teilnehmer als unabhängige Knoten in der Lage sind, miteinander zu kommunizieren.

    Die Vernetzung zweier Knoten in einem ITS kann anhand der Art der Knoten, die miteinander kommunizieren, kategorisiert werden.
    \newpage
    \subsection{Vehicle to Everyhing}
    Grundlage eines ITS ist die \textbf{Vehicle to Everything} Kommunikation, die das Fahrzeug als zentralen Punkt in den
    Kontext der unterschiedlichen Umgebungssysteme und Entitäten setzt. Dabei wird genauer unterschieden in die Teilbereiche:
    
    \subsubsection{Vehicle to Vehicle}
    \textbf{Vehicle to Vehicle} (V2V) beschreibt die Ad-Hoc Vernetzung mehrerer Fahrzeuge untereinander mit dem Ziel relevante Fahrzeugdaten z.Bsp. über Richtung und Geschwindigkeit auszutauschen.
    Desweiteren sollen über V2V Kommunikation weitere sicherheitsrelevante Nachrichten aus anderen Teilbereichen weitergeleitet werden.

    \subsubsection{Vehicle to Network}
    \textbf{Vehicle to Network} (V2N) beschreibt die Vernetzung des Fahrzeugs mit dem Telekommunikationsnetz und der Cloud. Dadurch können über den
    rein lokalen Kontext hinaus Ressourcen genutzt und Daten geteilt werden. Ein Beispiel für eine solche V2N Anwendung, die bereits heute
    im Einsatz ist, ist die Integration cloudbasierter Navigationslösungen wie Google Maps in das Fahrzeug.
    
    \subsubsection{Vehicle to Infrastructure}
    \textbf{Vehicle to Infrastructure} (V2I) (alternative Bezeichnung: Vehicle to Roadside) beschreibt die Vernetzung des Fahrzeugs mit der umgebenden Verkehrsinfrastruktur. 
    Zum Beispiel könnten Mautstationen automatisiert Fahrzeuge erfassen und abrechnen oder smarte Ampelanlagen könnten die Ampelzyklen in Abhängigkeit
    der Anzahl der jeweils wartenden Fahrzeuge anpassen, um den Verkehrsfluss zu optimieren. 

    \subsubsection{Vehicle to Pedestrian}
    \textbf{Vehicle to Pedestrian} (V2P) beschreibt die Vernetzung des Fahrzeugs mit nichtmotorisierten Verkehrsteilnehmern.
    Während bei den restlichen Vernetzungarten beide Endknoten als elektronische Systeme definitionsgemäß vernetzungsfähig sind, ist bidirektionale V2P Kommunikation nur möglich
    falls der nichtmotorisierte Verkehrsteilnehmer ein entsprechendes Gerät, zum Beispiel ein Smartphone mitführt, dass diese Funktionalität unterstützt.
    Daher ist V2P ist auch allgemeiner zu verstehen und umfasst neben der aktiven bidirektionalen Kommunikation der Entitäten auch die Erfassung von Fussgängern über rein
    fahrzeugseitige Sensorsysteme.

    Ziel von V2P ist explizit der Schutz der nichtmotorisierten Verkehrsteilnehmer, die bei Unfällen einen inhärenten Nachteil haben.

    \subsubsection{Vehicle to Device}
    \textbf{Vehicle to Device} (V2D) beschreibt die Kommunikation des Fahrzeugs mit elektronischen (Hand-)Geräten. Aktuelle Einsatzgebiete für diese Form der Kommunikation
    sind mobile Applikationen für Smartphones, mit denen Fahrzeugfunktionen von außerhalb gesteuert werden können. Beispiele hierzu
    sind die Smartphone Applikation von Tesla, mit der Fahrzeuge ausgeparkt werden können oder eine neue Entwicklung von Volvo mit der physische Schlüsseltechnologie
    durch eine mobile Smartphoneapp zuerst ergänzt und später ersetzt werden soll. ~\cite{BP05}
     
    \subsubsection{Vehicle to Grid}
    \textbf{Vehicle to Grid} (V2G) beschreibt die Vernetzung eines elektrischen Fahrzeuges mit dem Stromnetz mit dem Ziel die Batterien elektrischer Fahrzeuge als Speichermedien
    bidirektional in das Stromnetz zu integrieren. \cite{BP08}

    \subsection{ITS Technologie und Standardisierung}

    Aufgrund der heterogenen Landschaft an Fahrzeugherstellern, Infrastrukturbetreibern, Komponententechnologien und der Tatsache, dass Möglichkeiten
    und Beschränkungen dieser neuartigen Technologien noch nicht ausgelotet sind wurde bereits früh ein konkreter Standardisierungs- und Normierungsbedarf erkannt.

    Aus dieser Erkenntnis heraus sind die wichtigsten Standardisierungsorganisationen ISO, CEN, ETSI und SAE dabei Referenzarchitekturen und Protokolle zu entwickeln, die
    die Grundlagen der Weiterentwicklung der Technologie in die Zukunft bilden sollen.\cite{BP11}\\
    Da die Lebens- und Nutzungsspanne von Fahrzeugen mittlerweile bei 15 Jahren angelangt ist und in naher Zukunft bis zu 20 Jahren prognostiziert werden, während gleichzeitig die 
    Lebensspanne von Kommunikationsmedien immer kürzer wird liegt die grundlegende Schwierigkeit der Standardisierungsbemühungen darin, zukunftssichere Standards und Protkolle zu entwickeln,
    die von konkret verwendeten Kommunikationsmedien abstrahieren und sicherstellen, dass Fahrzeuge, die heute produziert werden auch in 20 Jahren noch ohne nennenswerte Einbussen der Funktionalität in die Verkehrssysteme
    integriert werden können. Da es für die Hersteller zudem nicht kosteneffizient ist, Fahrzeuge für regionale Märkte zu produzieren, müssen die Lösungen für ITS Systeme
    außerdem globale Einsatzmöglichkeiten bieten.\\
    Mit diesem Ziel hat die ISO mit der ISO 21217 \cite{BP09} bereits im Jahr 2010 die CALM Referenzarchitektur konkretisiert, die eine schichtbasierte Lösung präsentiert, die die Applikationsschicht, in
    der die intelligenten Transportsysysteme ihre Dienste bereitstellen und nutzen von der medienbasierten Kommunikation trennt.\\
    Die CALM Architektur bildet die Grundlage aller internationalen Standardisierungsbemühungen.

       
    \begin{figure}[h!]
        \begin{center}
        \includegraphics[width=0.7\linewidth]{./images/BP/CALM2.jpg}
        \caption[www.researchgate.net/figure/ETSI-ITS-G5-Architecture-from-15\textunderscore fig3\textunderscore 327727285]{CALM Architektur}
        \label{fig:CALM}
    \end{center}
    \end{figure}

    Aufbauend auf dieser Referenzarchitektur existieren spezifische Konkretisierungen für die verschiedensten drahtlosen und drahtgebundenen Kommunikationsmedien.
    Die zwei wichtigsten Spezifikationen sind dabei die drahtlose Kommunikation auf der Grundlage von WLAN, die aufbauend auf der IEEE 802.11a Spezifikation im IEEE 802.11p Protokoll \cite{BP10} spezifiert ist 
    und das 3GPP Protokoll für eine drahtlose Kommunikation auf Basis von Mobilfunktechnologie.

    \subsubsection{IEEE 802.11p}
    Der IEEE 802.11p Standard, spezifiziert eine Erweiterung des IEEE 802.11 WLAN Standards, der die drahtlose Kommunikation in und mit Fahrzeugen ermöglicht.

    Diese Erweiterung war notwendig, da die Zeitspanne zur Kommunikation eines Fahrzeugs in Bewegung mit stationären Infrastrukturknoten meist nur sehr kurz ist, was prohibitiv für die komplexe Verbindungsaufbauphase
    im herkömmlichen WLAN ist. Daher ermöglicht 802.11p die sofortige Kommunikation der Teilnehmer, ohne zuvor ein Basic Service Set zu etablieren.\\
    Da dies jedoch bedeutet dass die kommunizierenden Stationen nicht assoziert und nicht authentifiziert sind, müssen entsprechende Sicherheitsmechanismen in höheren Netzwerkschichten
    implementiert werden.

    IEEE 802.11p bildet den Grundstein für die Dedicated Short Range Communication Technik (DSRC), die in Europa als ITS-G5 bezeichnet wird. 
    
    Neben der drahtlosen Kommunikation auf Grundlage von WLAN, ist besonders für die Vernetzung über längere Entfernung die Verwendung bestehender oder zukünftiger Mobilfunksysteme möglich.
    Diese Kommunikationstechnologien werden in der 3GPP Protokollfamilie spezifiziert.

   
    \subsection{Zusammenfassung und Fazit}
    Die Einbindung immer komplexerer elektronischer Systeme in Fahrzeugen hat die Entwicklung in den vergangenen
    20 Jahren rasant vorangetrieben. Zum Teil veraltete Standards müssen überdacht und modernisiert werden.\\
    Entwicklungen in der nahen Zukunft, die das Fahrzeug aus dem individuellen Kontext herauslösen und in ein holistisches volllvernetztes
    System integrieren mit dem Ziel erhöhter Sicherheit, verminderter Auswirkungen der Mobilität auf die Umwelt und der Entwicklung neuer Geschäftsmodelle
    und Mobilitätskonzepte stellen eine Herausforderung an die Standardisierungsorganisationen, da die Zusammenführung und Kooperation heterogener Systeme zu einem
    vereinheitlichten Gesamtsystem nur auf Grundlage zukunftsfähiger Standards möglich ist.\\
    
    Zudem bedeuten neue Antriebskonzepte wie Hybridmotoren oder rein elektrische Antriebe, dass sich die Landschaft elektronischer Fahrzeugsysteme bereits kurzfristig 
    weiter verändern wird. Viele Sensorsysteme zur Überwachung der mechanischen Motorkomponenten werden in elektrisch betriebenen Fahrzeugen obsolet, während
    neue komplexe Sensorsysteme, die teil- und vollautonomes Fahren ermöglichen, zunehmend Einzug in die Fahrzeuge halten werden.\\

    Daher kann und soll diese Ausarbeitung nur ein Ausgangspunkt für eine weitere Erschliessung des Themas in der gesamten Komplexität darstellen.


\bibliography{References}
\bibliographystyle{abbrv}
\listoffigures

\end{document}
