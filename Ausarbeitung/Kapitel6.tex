\section{ECU / Steuergeräte}
    \subsection{Einführung}
    Durch die immer neuen Fortschritte in der Technologie, werden die früher mechanisch realisierten
    Funktionen heutzutage elektronisch umgesetzt. Hierzu werden die "electronic control units"
    geschaffen. Mit "electronic control unit (ECU)" wird jedes Embedded System in einem
    Automobil gemeint. Dieses System kontrolliert jegliche elektrische Systeme
    oder Subsysteme im ganzen Fahrzeug, es ist sozusagen das Herzstück. Das ECU gibt
    Instruktionen und Anweisungen für viele Variaten von elektrischen System. Es übermittelt
    diesen Systemen Instruktionen, wie die einzelnen Systeme zu operieren beziehungsweise
    zu funktionieren haben. Neue Fahrzeug könne bis zu 80 ECUs besitzen, dies erhöht die
    Komplexität und dazugehörige Programmierarbeit für das Zusammenspiel aller ECUs. Ein
    Paar wichtige Steuergeräte sind hier zum Beispiel das BCM, PCM, GEM und die ECU.
    Um die ECUs vor ungewollter Korruption zu bewahren, werden diese geschützt. 
    ~\cite{ECU.PB6} ~\cite{ECU.PB5} ~\cite{ECU.PB4} ~\cite{ECU.PB3} ~\cite{ECU.PB2} ~\cite{ECU.PB1}

    \subsection{Typen}
        \subsubsection{Brake Control Module (BCM)}
        Zu dem Brake Control Module gehören Systeme wie ABS, TCS und ESC/ESP. Während diese Systeme früher
        nur in Luxusfahrzeugen zu finden waren, sind sie heute in fast jeder Fahrzeugklasse Standard.\\
        Diese Systeme kontrollieren, wie der Name erkennen lässt, die Steuerung der Bremse.

        \begin{itemize}
            \item \textbf{ABS} (Antiblockiersystem) verhindert das bei einer Vollbremsung die Räder
            blockiert werden und der Fahrer die Kontroller über das Fahrzeug verliert.
            ~\cite{antiblockiersys.PB1}

            \item \textbf{TCS} (Traction control System), ist für die Regelung der Räder bei rutschiger oder glatter Fahrbahn zuständig
            und soll das Durchdrehen der Räder verhindern, damit der Fahrer auch bei unvorteilhaften Umweltbedingungen ein
            Maximu an Fahrzeugkontrolle hat.
            ~\cite{tractioncontrol.PB1}

            \item \textbf{ESC} (Electronic stability control) / \textbf{ESP} (Elektronisches Stabilitätsprogramm)
            ist ein Sicherheitssystem zur Spur- und Stabilitätskontrolle des Fahrzeugs.
            Es soll als Unterstützung bei Ausweichmanövern oder beim Verlust der Kontrolle über das Fahrzeug dienen. 
            Zu diesem Zweck wird das Motormomentum reduziert und sollte dies nicht ausreichen leitet das System einen aktiven
            Bremsvorgang ein.
            ~\cite{ESP.PB1}
        \end{itemize}

        \subsubsection{Engine Control Unit (ECU)}
        Die ECU ist das zentrale Element des Motormanagement System bzw. des Motorsteuergeräts. Es verwaltet alle
        Komponenten, die für den Motor von Bedeutung sind, so zum Beispiel die Kontrolle des Brennstoff- und Luftzuflusses und die Zündung.
        ~\cite{enginecontrol.PB1} 

        \subsubsection{Powertrain Control Module (PCM)}
        Das Powertrain Control Module ist vergleichbar mit dem Gehirn des Menschen. Das Gehirn bei einem Menschen
        koordiniert die einzelnen Funktionen der verschiedenen Körperteile, sodass alle im Einklang und ohne
        Störungen kooperieren. Nach dem gleichen Prinzip übernimmt das Powertrain Control Module im Fahrzeug die Koordination der einzelnen
        Subsysteme für eine reibungslose Zusammenarbeit.\\ Diese reibungslose Arbeit kann das Powertrain
        Control Module durch die zahlreichen im Auto verbauten Sensoren bewerkstelligen.
        ~\cite{PCM.PB1}

        \subsubsection{Door Control Unit (DCU)}
        Die DCU befindet sich in der Innenseite der Fahrer- und Beifahrertüren und ist verantwortlich für die
        Kontrolle der Komponenten in der Tür, zum Beispiel der Zentralverriegelung, dem Fensterheber und der Seitenspiegelanpassung. 
        Vereinfacht kann gesagt werden, dass alle türbezogenen Funktionen durch die Door Control Unit gesteuert werden.
        ~\cite{doorcontrol.PB1} ~\cite{doorcontrol.PB2}

        \subsubsection{Electric Power Steering Control Unit (PSCU)}
        Die PSCU sammelt Daten von den einzelnen EPS ("electronic power steering") Komponenten. Diese umfassenen EPS Motor,
        Getriebe und den Torque Sensor.\\ Durch dieAuswertung dieser Informationen kann das PSCU die Lenkaufgabe des
        Fahrers aktiv unterstützen. Die PSCU arbeitet unabhängig vom Motor und funktioniert somit auch bei ausgeschaltetem Motor.
        ~\cite{PSCU.PB1}

        \subsubsection{Human-Machine Interface (HMI)}
        Die HMI Unit ist verantwortlich für die Bedienung und Interaktion zwischen Insassen und Fahrzeug. Sie besteht aus einer
        Hardware- und Softwarekomponente, die Eingaben von dem Benutzer in Signale umwandelt, die als Daten von weiteren Fahrzeugkomponenten
        verarbeitet werden können.
        ~\cite{HMI.PB1} ~\cite{HMI.PB2}

        \subsubsection{Seat Control Unit}
        Durch neue Technologie und Fortschritt haben neue oder luxus Autos ein Steuergerät für die Funktionen der Sitzeinstellungen.
        Durch dieses Gerät kann eine vielzahl von Positionen eingestellt und kleine Anpassungen vorgenommen werden.
        Der Gedanke hinter dieser Einheit, ist zum einen der Komfort der eine große Rolle spielt und zum anderen die Sicherheit,
        die durch die richtigen Einstellungen gewährleistet wird. Beispiele für die Einstelungen sind die Winkel der Sitzlehne,
        Sitzhöhe oder Sitzweite.
        ~\cite{seatcontrol.PB1} ~\cite{seatcontrol.PB2} 

        \subsubsection{Speed Control Unit (SCU)}
        Die Speed Control Unit oder auch Speed Control System besteht aus mehreren Teilkomponenten, die in der Zusammenarbeit
        verschiedene Aktionen ausführen können. Aktionen die durch Daten von bzw. für das Speed Control System ausgeführt
        werden können sind das halten der Geschwindigkeit bei jedem Gelände oder das überschreiten einer zuvor festgelegten
        Geschwindigkeit. Bei beiden Aktionen werden Daten an das System gesendet und von diesem verarbeitet.
        ~\cite{SCU.PB1}

        \subsubsection{Telematic Control Unit (TCU)}
        Die TCU ist ein Steuergerät, mit die Position des Fahrzeugs bestimmt werden kann. Es kombiniert das Überwachungssystem,
        das Tracking System und die WLAN Kommunikation. Die Telematic Control Unit hat eine Entwicklung von der Bearbeitung, von
        Telekommunikation und Informationen hin zum Verbinden der GPS-Daten und WLAN Kommunikationen für die Unterstützung.
        Diese Einheit kann unter anderem benutzt werden, um Informatinen über das Radio oder das GSM module zu erhalten.
        Dieses Steuergerät bietet auch weiter wichtige Funktionen in den neuen Autosmobilen, hier gibt es die Verbindung
        des Autos zu der Cloud, Fahrer und Mitfahrer sicher zu halten und den Verkehrfluss zu optimieren.
        ~\cite{telematiccontrol.PB1} ~\cite{telematiccontrol.PB2} ~\cite{telematiccontrol.PB3}

        \subsubsection{Transmission Control Unit (TCU)}
        Die Transmission Control Unit kontrolliert die Schaltvorgänge im Getriebe. Eine Software kontrolliert das Schalten
        zwischen den einzelnen Gängen und nimmt im gleichen Zug Anpassungen am Schaltverhalten vor, wobei diese Funktionalität auf Automatikgetriebe beschränkt ist. Diese Funktion kann
        Durch die Integration weiterer Komponenten kann die Transmission Control Unit das Fahrverhalten eines Fahrzeugs verbessern.\\
        Zudem kann das Steuergerät das Schaltverhalten anpassen, falls ein Fahrer die Möglichkeit hat zwischen sportlichen und normalem Fahreinstellungen zu wählen.
        ~\cite{transmissioncontrol.PB1} ~\cite{transmissioncontrol.PB2}

        \subsubsection{Battery Management System (BMS)}
        Das BMS verwaltet wiederaufladbare Batterien, damit diese bemi Be- und Entladen nicht beschädigt werden und außerhalb ihrer
        Sicherheitszone arbeiten. Im Zusammenhang mit der Automobilbranche muss das BMS mit verschiedenen Komponenten
        zusammenarbeiten und in Echtzeit regelnd eingreifen können. Hier sind immer schnelle
        Ladungen und Entladungen durch das Bremsen und Beschleunigen vorhanden. Das BMS ist hier, wie erwähnt für die
        Gewährleistung der Sicherheit der Batterien.
        ~\cite{BMS.PB1} ~\cite{BMS.PB2}

        \subsubsection{Suspension Control Module (SCM)}
        Das SCM ist verantwortlich für die richtige Einstellung der Stoßdämpfer im Fahrzeug. Es sammelt Daten und Informationen
        passt die Einstellungen den Vorgaben gemäß an. Unter anderem soll dieses Modul das Fahrverhalten verbessern oder
        die aktuelle Karroseriehöhe insbesondere in Kurven beibehalten.
        ~\cite{suspensioncontrol.PB1}

        \subsubsection{Body Control Module}
        Das Body Control Module ist zuständig für den elektrischen Zugriff, komfort- und sicherheit Features. Es wird in
        den Nutzfahrzeugen für Beleuchtung, akustische Signale oder Scheibenwischer verwendet.
        ~\cite{BCM.PB1}