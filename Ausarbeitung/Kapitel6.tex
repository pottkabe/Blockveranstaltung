\section{ECU / Steuergeräte}
    \subsection{Einführung}
    Durch die immer neuen Fortschritte in der Technologie, werden die früher mechanisch realisierten
    Funktionen heutzutage elektronisch umgesetzt. Hierzu werden die electronic control units (ECU)
    geschaffen. Mit ECU wird jedes embedded System in einem
    Automobil bzeichnet. Dieses System kontrolliert jegliche elektrische Systeme
    oder Subsysteme in einem Fahrzeug. Das ECU gibt
    Instruktionen und Anweisungen für viele Varianten von elektrischen System. Es übermittelt
    diesen Systemen Instruktionen, wie die einzelnen Systeme zu 
    zu funktionieren haben. Neue Fahrzeug können bis zu 80 ECUs besitzen, dies erhöht die
    Komplexität und dazugehörige Programmierarbeit für das Zusammenspiel aller ECUs. Ein
    Paar wichtige Steuergeräte sind hier zum Beispiel das BCM, PCM, GEM und die ECU.
    ~\cite{ECU.PB6} ~\cite{ECU.PB5} ~\cite{ECU.PB4} ~\cite{ECU.PB3} ~\cite{ECU.PB2} ~\cite{ECU.PB1}

    \subsection{Typen}
        \subsubsection{Brake Control Module (BCM)}
       Das Brake Control Module kontrolliert die Steuerung der Bremse und umfasst Systeme wie Antiblockiersystem, Traction Control System und eltronischem
       Stabilitätsprogramm (Näheres hierzu im Kapitel "Fahrassistenzsysteme"). 
        
        Waren diese früher nur in hochpreisigen Fahrzeugen zu finden, sind sie Standard in fast jedem Fahrzeug.\\
        Diese Systeme kontrollieren wie der Name erkennen lässt, die Steuerung der Bremse.

        \subsubsection{Engine Control Unit (ECU)}
        Die Engine Control Unit ist das zentrale Element des Motormanagement Systems. Es verwaltet alle
        Komponenten, die für den Motor von Bedeutung sind, zum Beispiel die Kontrolle des Brennstoff- und Luftzuflusses und die Zündung.
        \cite{enginecontrol.PB1} 

        \subsubsection{Powertrain Control Module (PCM)}
        Das Powertrain Control Module koordiniert die verschiedenen Subsysteme des Automobil, damit diese störungsfrei zusammenarbeiten können.
        \cite{PCM.PB1}

        \subsubsection{Door Control Unit (DCU)}
        Die DCU befindet sich in der Innenseite der Fahrer- und Beifahrertüren und ist verantwortlich für die
        Kontrolle der Türkomponenten. Sie kontrolliert zum Beispiel die Zentralverriegelung, den Fensterheber und die Seitenspiegelanpassung. 
        Vereinfacht kann gesagt werden, dass alle türbezogenen Funktionen durch die Door Control Unit gesteuert werden.
        ~\cite{doorcontrol.PB1} ~\cite{doorcontrol.PB2}

        \subsubsection{Electric Power Steering Control Unit (PSCU)}
        Die PSCU sammelt Daten von den einzelnen EPS (Electronic Power Steering) Komponenten. Diese umfassenen EPS Motor,
        Getriebe und den Drehmomentsensor.\\ Durch die Auswertung dieser Informationen kann das PSCU die Lenkaufgabe des
        Fahrers aktiv unterstützen. Es arbeitet unabhängig vom Motor und funktioniert daher auch bei ausgeschaltetem Motor.
        \cite{PSCU.PB1}

        \subsubsection{Human-Machine Interface (HMI)}
        Die HMI Unit ist verantwortlich für die Bedienung durch und die Interaktion mit den Insassen. Sie besteht aus einer
        Hardware- und Softwarekomponente, die Eingaben von dem Benutzer in Signale umwandelt, die als Daten von weiteren Fahrzeugkomponenten
        verarbeitet werden können.
        \cite{HMI.PB1} \cite{HMI.PB2}

        \subsubsection{Seat Control Unit}
        In modernen Fahrzeuge ist meistens ein Steuergerät, das die Funktionen der Sitzeinstellungen steuert, verbaut.
        So kann eine Vielzahl an Sitzpositionen eingestellt und gespeichert werden, sowie kleine Anpassungen vorgenommen werden.
        Neben dem Komfortgewinn bieten diese Funktionen erhöhte Sicherheit für den Fahrer, da so eine korrekte Sitzposition insbesondere in
        Unfallsituationen gewährleistet werden kann.
        \cite{seatcontrol.PB1} \cite{seatcontrol.PB2} 

        \subsubsection{Speed Control Unit (SCU)}
        Die Speed Control Unit besteht aus mehreren Komponenten, die in der Zusammenarbeit
        verschiedene Aktionen ausführen können. Aktionen die durch Daten von bzw. für das Speed Control System ausgeführt
        werden können sind das Halten der Geschwindigkeit bei jedem Gelände oder das Verhindern des Überschreitens einer zuvor festgelegten
        Geschwindigkeit.
        \cite{SCU.PB1}

        \subsubsection{Telematic Control Unit (TCU)}
        Die TCU ist ein System zur Positionsbestimmung des Fahrzeugs. 
        Es kombiniert das Überwachungssystem, Tracking System und die WLAN Kommunikation. 
        Nachdem die TCU ursprünglich für die Verarbeitung der Telekommunikationsdaten eingesetzt wurde, 
        verbindet sie heute GPS-Daten und WLAN-Kommunikation.
        Die TCU ist ein zentrales System für viele moderne oder zukünftige Sicherheits und Komfortfunktionalitäten. Zum Beispiel
        können in Notfallsituationen Einsatzkräfte mobilisiert, automatische Mauterfassung durchgeführt oder mittels Vehicle to Vehicle
        Kommunikation der Verkehrsfluss optimiert werden. 
        \cite{telematiccontrol.PB1} \cite{telematiccontrol.PB2} \cite{telematiccontrol.PB3}

        \subsubsection{Transmission Control Unit (TCU)}
        Die Transmission Control Unit kontrolliert die Schaltvorgänge im Getriebe. Eine Software kontrolliert das Schalten
        zwischen den einzelnen Gängen und nimmt im gleichen Zug Anpassungen am Schaltverhalten vor, 
        wobei diese Funktionalität auf Automatikgetriebe beschränkt ist. Dadurch kann
         das Fahrverhalten eines Fahrzeugs verbessert werden.\\
        Zudem kann das Steuergerät das Schaltverhalten anpassen, falls ein Fahrer die Möglichkeit
         hat zwischen sportlichem und normalem Fahreinstellungen zu wählen.
        \cite{transmissioncontrol.PB1} \cite{transmissioncontrol.PB2}

        \subsubsection{Battery Management System (BMS)}
        Ein BMS steuert wiederaufladbare Batterien, damit diese nicht beschädigt werden und außerhalb ihrer
        Sicherheitszone arbeiten. Im Kontext eines Kraftfahrzeuges muss das BMS mit verschiedenen Komponenten
        zusammenarbeiten. Eine weitere Herausforderung ist die Echtzeitfunktionalität des BMS, da es zu sehr kurzen (Ent-)Ladezyklen durch Beschleunigungs-
        und Bremsvorgänge kommen kann.
        \cite{BMS.PB1} \cite{BMS.PB2}

        \subsubsection{Suspension Control Module (SCM)}
        Das SCM ist verantwortlich für die richtige Einstellung der Stoßdämpfer im Fahrzeug. Es sammelt Daten und Informationen und
        passt die Einstellungen den Vorgaben gemäß an. Unter anderem soll dieses Modul das Fahrverhalten verbessern oder
        die aktuelle Karroseriehöhe insbesondere in Kurven beibehalten.
        \cite{suspensioncontrol.PB1}

        \subsubsection{Body Control Module}
        Das Body Control Module ist zuständig für den elektronischen Zugriff, Komfort- und Sicherheit Features. Es wird in
         Nutzfahrzeugen für Beleuchtung, akustische Signale oder Scheibenwischer verwendet.
        \cite{BCM.PB1}