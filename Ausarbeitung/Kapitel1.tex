\section{Abstrakt}
Mittlerweile machen elektronische Systeme etwa ein Drittel der Gesamtkosten bei der Produktion von
Personenkraftwagen aus ~\cite{.BP02}. Von Motorsteuerung, über aktive und passive Sicherheitssysteme, 
Wartung und Diagnose bis hin zur Unterhaltungselektronik sind Personenkraftwagen inzwischen hochgradig
vernetzte Systeme.\\
Mit den aktuellen Entwicklungen in Richtung teil- und vollautonohmer Systeme wird diese Vernetzung noch weiter zunehmen 
und die elektronischen Systeme werden der Hauptwertträger eines Fahrzeugs werden.\\

Ziel dieser Ausarbeitung ist es dem interessierten Leser einen Überblick über die wichtigsten elektronischen Systeme in
modernen Fahrzeugen und deren Interaktion untereinander zu geben. Ein gewisses technisches Grundverständnis vorrausgesetzt 
soll er in der Lage sein, neue Entwicklungen in den Kontext des aktuellen Stand der Technik zu setzen.\\

Da es sich um ein komplexes Thema handelt, dass auf beschränktem Platz dargeboten werden soll, müssen gewisse Teilbereiche naturgemäß
kürzer ausfallen oder gänzlich ignoriert werden.  